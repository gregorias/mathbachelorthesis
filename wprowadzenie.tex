\chapter{Wprowadzenie}

Teoria ciał sprawnie udowodniła niemożliwość rozwiązania odwiecznych problemów
geometrycznych takich jak kwadratura koła czy trysekcja kąta. W 1986 White
\cite{whi88} dodał kolejny naturalny przykład mocy teorii ciał. Udowodnił, że
nietrywialne przekształcenie postaci
\[
  \left( \ldots \left(\left(x + a_1\right)^{q_1} + a_2\right)^{q_2} +
    \ldots + a_n\right)^{q_n}, a_n \in Z, q_n = p^{k_i}, k_i \in \Z
\]
nie może być równe $x$.

W 1991 Cohen \cite{coh91} uogólnił wynik White'a na wymierne potęgi, gdzie
wykładniki są ułamkami nieparzystych liczb pierwszych. W 1995 Cohen i Glass
\cite{coh95} przedstawili znacznie uproszczony dowód twierdzenia White'a, który
korzysta tylko z elementarnych faktów teorii ciał i nie korzysta z teorii
Galois.

Ta praca przedstawia polski przekład dowodu Cohena i Glassa wraz z dodatkowymi
wyjaśnieniami. Twierdzenie White'a formalnie dotyczy pokazania, że grupa
generowana przez translację i potęgi jest wolna, dlatego rozdział
\ref{ch:free_groups} przedstawia podstawowe definicje i fakty dotyczące grup
wolnych. Rozdział \ref{ch:auxiliary} przedstawia dowody lematów i twierdzeń
wykorzystywanych w dowodzie głównego twierdzenia, w szczególności fundamentalny
dla twierdzenia lemat o niemożliwości zachodzenia pewnej identyczności wśród
słów wielomianowych. Dowody w tym rozdziale można pominąć przy pierwszym
czytaniu. W rozdziale \ref{ch:white_theorem} jest zaprezentowana główna teza
wraz z jej dowodem i intuicjami.
