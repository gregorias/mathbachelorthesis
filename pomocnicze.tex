\chapter{Pomocnicze lematy i twierdzenia}
\label{ch:auxiliary}
\section{Nietrywialne pierwiastki wielomianów nie należą do $\overline{\Q}(x)$}
\begin{lemma}
  Niech $a \in \overline{\Q}, a \neq 0, f(x) \in \overline{\Q}[x], \deg f(x)
  \neq 0$ i $p \in \N$ pierwsza.  Wówczas nie istnieje taka funkcja wymierna
  $w(x) \in \overline{\Q}(x)$, że $f(x)^{p} + a = w(x)^p$. Innymi słowy
  wielomiany niebędące trywialnie $p$-tymi potęgami nie posiadają $p$-tego
  pierwiastka w ciele funkcji wymiernych.
  \label{lem:nontrivial_roots}
\end{lemma}

\begin{proof}
Załóżmy przeciwnie.

Jeśli $w(x) = \frac{m(x)}{n(x)}$, gdzie $m(x), n(x) \in \overline{\Q}[x], \deg
n(x) \neq 0$ i jest to ułamek nieskracalny, to w równaniu: $f(x)^{p} + a =
\frac{m(x)^p}{n(x)^p}$ po lewej stronie mamy wielomian. Zatem $n(x)^p | m(x)^p$,
ale założyliśmy, ze $n(x) \! \not| m(x)$. Sprzeczność.

Zatem $w(x) = g(x), g(x) \in \overline{\Q}[x]$. Mamy:
\begin{eqnarray*}
  f(x)^p + a &=& g(x)^p \\
  a &=& g(x)^p - f(x)^p\\
  a &=& \left(g(x) - f(x)\right)\left(\sum_{i=0}^{p-1}g(x)^if(x)^{p-1 - i}\right)\\
\end{eqnarray*}

Możemy założyć, że $f(x)$ jest wielomianem monicznym. Z końcowego równania $g(x)
- f(x)$ dzieli $a$, zatem $g, f$ są tego samego stopnia. Zatem w prawym
czynniku mamy wielomian stopnia $(p-1)\deg g(x)$ co jest większe od zera.
Sprzeczność.
\end{proof}

\section{Lematy o słowach wielomianowych}
\begin{defin}[Słowo wielomianowe]
  Niech $p$ będzie ustaloną nieparzystą liczbą pierwszą. Wielomian postaci

  \begin{equation}
    \left( \ldots \left(\left(x + a_1\right)^{q_1} + a_2\right)^{q_2} +
      \ldots + a_n\right)^{q_n}, a_n \in Z, q_n = p^{k_i}, k_i \in \N
  \label{eq:polynomial_word_extended}
  \end{equation}

  nazywamy słowem wielomianowym.
\end{defin}

Nazwa słowo wielomianowe bierze się stąd, że ten wielomian można przedstawić w
postaci iloczynu $t^{a_1}e_p^{k_1} \ldots t^{a_n}e_p^{k_n}$, gdzie $t, e_p \in
\Sigma_{\R}, t(x) = x + 1, e_p(x) = x^p$. W rozdziale \ref{ch:white_theorem}
pokażemy, że jest to grupa wolna zatem ta reprezentacja stanowi słowo
zredukowe.

\begin{lemma}
  Każdy pierwiastek słowa wielomianowego $f$ jest pierwiastkiem o krotności
  będącej potęgą $p$.
  \label{lem:multiplicity_lemma}
\end{lemma}

\begin{proof}
Definiujemy ciąg słów wielomianowych $f_1, f_2, \ldots, f_{l+1}$ jako ciąg
kolejnych aplikacji funkcji postaci $(x+a_i)$ ze wzoru
\ref{eq:polynomial_word_extended}.
To jest
\[f_1(x) = x + a_1, f_2(x) = f_1^{q_1}(x) + a_2, \ldots, f_{l+1}(x) =
f_l^{q_l}(x) = f(x)\]

Różniczkując otrzymujemy:
\[f' = \prod_{i=1}^l q_i f_i^{q_i - 1}\]

Zatem każdy wielokrotny pierwiastek $f$ jest także pierwiastkiem co najmniej
jednego z wielomianów $f_j$. Niech $f_{j_1}, \ldots, f_{j_s}$, gdzie $1 \leq j_1
< \ldots < j_s \leq l$ będą wszystkimi wyrazami z ciągu, dla których dany
pierwiastek $f$ jest również i ich pierwiastkiem. Wówczas stopień tego
pierwiastka to $\prod_{i=1}^s q_{j_i}$.
\end{proof}

\begin{lemma}
  Niech $f$ będzie słowem wielomianowym ($a_1, \ldots, a_{l+1} \neq 0$) i
  $\omega$ ($\neq 1$) będzie $p$-tym pierwiastkiem jedności w $\overline{\Q}$.
  Wówczas identyczność postaci
  \begin{equation}
    f(x)f^t(\omega x)g\left(x^p\right) = h^p(x)
    \label{eq:polynomial_lemma}
  \end{equation}

  gdzie $t \in \Z$, a $g, h \in \overline{Q}[x]$ jest niemożliwa.
  \label{lem:main_polynomial_lemma}
\end{lemma}

[TODO dodaj wyjaśnienie czemu to twierdzenie jest pożyteczne]

\begin{proof}
  Jeśli \ref{eq:polynomial_lemma} zachodzi to możemy założyć, że $0 \leq t \leq p - 1$
  poprzez włączenie $f^p(\omega x)$ do $h^p(x)$.

  Dodatkowo istnieją $g'(x), h'(x) \in \overline{\Q}[x]$ takie, że
  \begin{equation}
    f(x) f^t(\omega x) = g' \left( x^p \right) h'^p(x)
    \label{eq:polynomial_lemma_right}
  \end{equation}
  Tak jest, gdyż zauważmy, że
  \[ f(x) f^t(\omega x) = \frac{h^p(x)}{g \left( x^p \right)} \]
  Po lewej stronie równania mamy wielomian, zatem $g \left( x^p \right) |
  h^p(x)$. Wielomiany $g', h'$ można skonstruować indukcyjnie następującą
  metodą. Niech dany będzie wielomian postaci
  $\frac{n_l \left( x^p \right) m^p(x)}{n_m \left( x^p \right)}$, gdzie $n_l,
  n_m, m$ to wielomiany w $\overline{\Q}$. Zauważmy, że $x^p - a$ ma wszystkie
  $p$ pierwiastków różne i dla różnych $a$ te wielomiany nie mają wspólnych
  pierwiastków. Jeśli $n_m \equiv 1$ to nic nie robimy, bo wtedy $g' = n_l, h' =
  m$.
  Jeśli $\left(x^p - a \right) | n_m \left(x^p\right)$ i $\left(x^p - a \right)
  | n_l \left(x^p\right)$ to bierzemy
  $n_l' = \frac{n_l(x)}{x^p - a}, n_m' = \frac{n_m(x)}{x^p - a}$ i w kolejnym
  kroku podstawiamy te skrócone wielomiany.
  W przeciwnym wypadku
  $\left(x^p - a \right) | n_m \left(x^p \right)$ i
  $\left(x^p - a \right) | h^p \left(x \right)$. Wszystkie pierwiastki
  $\left(x^p - a \right)$ są różne zatem $\left(x^p - a \right) | h(x)$.
  Bierzemy zatem:
  $n_l' = n_l(x)\left(x^p - a \right)^{p-1}, n_m' = \frac{n_m(x)}{x^p -
  a}, h' = \frac{h(x)}{x^p - a}$.
  W każdym kroku skracamy mianownik z kolejnych dzielników zatem w końcu
  uzyskamy żądaną postać $g'\left(x^p \right) h'^p(x)$.

  Niech $\alpha$ będzie jednokrotnym pierwiastkiem $f$. Wówczas każdy
  pierwiastek $\beta$ wielomianu $\left(x + a_1\right)^{q_1} - \left(\alpha +
  a_1\right)^{q_1}$ jest również jednokrotnym pierwiastkiem $f$. Tak jest, bo
  wówczas $f_1(\beta) = f_1(\alpha)$, a $f_1(\alpha)$ jest jednokrotnym
  pierwiastkiem $f \circ f_1^{-1}$.

  Niech $f^*(x)$ będzie iloczynem czynników $(x-\alpha)$, gdzie $\alpha$ jest
  jednokrotnym pierwiastkiem $f$. Wówczas $f^*(x) = f_0 \left( \left(x + a_1
  \right)^{q_1} \right)$ [TODO czemu wywalanie pierwiastków wielokrotnych nie
  psuje nam tego?] dla pewnego wielomianiu $f_0$.

  Jeśli $t=0$ to z lematu \ref{lem:multiplicity_lemma} dla pewnego wielomianu
  $g_0$
  \begin{equation}
   f^*(x) = f_0 \left(\left(x + a_1 \right)^{q_1} \right) = g_0 \left( x^p
  \right)
  \label{eq:polynomial_lemma_only_g}
\end{equation}
  Jeśli oznaczymy $N = \deg f^*$ to $f_0 \left(\left(x + a_1 \right)^{q_1}
  \right)$ musi mieć niezerowy współczynnik przy $x^{N-1}$, ale prawa strona z
  kolei nie może mięć niezerowego współczynnika przy $x^{N-1}$, bo $p$ wyznacza
  nam różnicę stopni między kolejnymi jednomianami. Zatem $t \neq 0$.

  W \ref{eq:polynomial_lemma} możemy zastąpić $f$ przez $f^*$ poprzez włączenie
  wielokrotnych pierwiastków pod $h'^p(x)$. Zauważmy, że po takim zastąpieniu
  mamy $h' \equiv 1$, gdyż nie mamy już $p$-krotnych pierwiastków po lewej
  stronie identyczności. Jeśli $\alpha$ jest pierwiastkiem $f^*$ to oznaczmy
  krotność $\omega^i\alpha$, gdzie $0 \leq i \leq p - 1$, jako $m_i$. Mamy
  $\forall_{1 \leq i \leq p - 1} \, m_i \in \{0, 1\}, m_0 = 1$.  Każda liczba
  postaci $\omega^i\alpha$ ma tę samą krotność jako pierwiastek $g' \left( x^p
  \right)$. Zatem
  \begin{align*}
    m_0 + tm_{p-1} &\equiv m_1 + tm_0 \pmod{p}\\
    m_1 + tm_{0} &\equiv m_2 + tm_1 \pmod{p}\\
                 &\mathrel{\makebox[\widthof{=}]{\vdots}}\\
    m_{p-2} + tm_{p-3} &\equiv m_{p-1} + tm_{p-2} \pmod{p}\\
  \end{align*}

  Zatem $\sum_{i=0}^{p-1} m_i \equiv 0 \pmod{p}$ lub $t \equiv p-1 \pmod{p}$. Z
  obu wynika, że $\forall_{0 \leq i \leq p-1} m_i = 1$. Wówczas $f(x) = f(\omega
  x)$, czyli $\left( f^*(x) \right)^t = g_0 \left( x^p \right)$ co jest
  niemożliwe używając analogicznego argumentu co przy
  \ref{eq:polynomial_lemma_only_g}.


\end{proof}

\section{Lematy i twierdzenia o rozszerzeniach ciał}
\begin{theorem}
  Niech $F$ będzie ciałem a $p$ liczbą pierwszą. Następujące fakty są
  równoważne.

  \begin{itemize}
    \item $x^p - u$ nie ma pierwiastka w $F$.
    \item $x^p - u$ jest nierozkładalne w $F$.
  \end{itemize}
  \label{th:14_1_1}
\end{theorem}

  Wynika wprost z \cite[Twierdzenie 14.1.1]{rom06}

\begin{theorem}
  Niech $F$ to ciało i $u \in F$.

  Jeśli $p$ jest nieparzystą liczbą pierwszą i $p \in \N$ to wielomian
  $x^{p^m} - u$ jest nierozkładalny w ciele $F$, jeśli
  $x^p - u$ nie ma pierwiastka w ciele $F$.
  \label{th:14_1_2}
\end{theorem}

  Wynika wprost z \cite[Twierdzenie 14.1.2]{rom06}

\begin{lemma}{Lemat o związku między prostymi rozszerzeniami}
  Niech $E$ będzie podciałem ciała $\C$ i $\overline{\Q} \subseteq E$.
  Niech $p$ będzie nieparzystą liczbą pierwszą.
  Niech $m, n \in \N$ takie, że $1 \leq m \leq n$ i $a, b \in E$ takie, że $x^p
  - a, x^p - b$ są nierozkładalne w $E$.
  Jeśli dla pewnych rozwiązań $x^{p^m} - a, x^{p^n} -b$ mamy
  $E\left(\sqrt[p^m]{a}\right) = E\left(\sqrt[p^n]{b}\right)$ to $ab^t =
  c^{p^m}$ dla pewnego $c \in E$ i liczby całkowitej $t$ niepodzielnej przez
  $p$.
  \label{lem:associated_extensions}
\end{lemma}
\begin{proof}
  Z \cite[Twierdzenie 14.2.8]{rom06} jeśli $x^{p^m} - a$ i $x^{p^n} - b$ są
  nierozkładalne to mają to samo rozszerzenie rozdzielcze nad $E$ wtedy i tylko
  wtedy, gdy istnieje $c \in E$ i $t \in \Z$ niepodzielne przez $p$ takie, że
  $a = c^nb^t$.
  Jako, iż $\overline{\Q} \subseteq E$ to jeśli $F$ rozszerzenie $E$ i $\alpha
  \in F$  jest pierwiastkiem $x^{p^m} -a$ to jeśli $\omega$ jest pierwotnym
  pierwiastkiem jedynki stopnia $p^m$ to $x - \omega^i \alpha, 0 \leq i \leq p^m
  - 1$ są wszystkimi czynnikami $x^{p^m} - a$.
  Wystarczy, więc pokazać, że $x^{p^m} - a$ i $x^{p^n} - b$ są nierozkładalne
  aby otrzemać tezę lematu.

  Nierozkładalność obu wielomianów wynika z twierdzeń \ref{th:14_1_1},
  \ref{th:14_1_2} i tego, że $x^p -a, x^p - b$ są nierozkładalne w $E$.
\end{proof}

\begin{theorem}
  Niech $p(x) \in F[x]$ to nierozkładalny w $F$ wielomian stopnia $n$.
  Niech $K = F[x]/p(x)$. Wówczas $[K:F] = n$
  \label{th:13_1_6}
\end{theorem}

  \cite[Twierdzenie 13.1.6]{dum04}

\begin{theorem}
  Niech $F, K$ to będą ciała takie, że $F \subseteq K$.
  Jeśli $\alpha \in K, \alpha \not \in F$ i dla pewnego wielomianu
  nierozkładalnego w $F$: $p(x) \in F[x]$ mamy $p(\alpha) = 0$ to
  $F(\alpha) = F[x]/p(x)$
  \label{th:13_1_4}
\end{theorem}

  \cite[Twierdzenie 13.1.4]{dum04}

\begin{lemma}
  Niech $E$ będzie podciałem $\C$. Niech $p$ będzie liczbą pierwszą, $a \in E$,
  $m \in N$ i $x^p - a$ będzie nierozkładalny w $E$.
  Jesli $\sqrt[p^m]{a}$ jest pewnym pierwiastkiem $x^{p^m} - a$ to
  $\left[E \left(\sqrt[p^m]{a} \right) : E \right] = p^m$.
\end{lemma}

  Widzieliśmy już, że $x^{p^m} - a$ musi być nierozkładalny. Zatem korzystając z
  twierdzeń \ref{th:13_1_4}, \ref{th:13_1_6} otrzymamy żądany rezultat.
