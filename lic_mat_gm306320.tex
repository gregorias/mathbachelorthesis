\documentclass{pracamgr}

\usepackage{polski}
\usepackage[utf8]{inputenc}

\usepackage{amsthm}
\usepackage{amsfonts}
\usepackage{amsmath}
\usepackage{amssymb}

\usepackage{calc}

\newcommand{\N}{\mathbb{N}}
\newcommand{\Z}{\mathbb{Z}}
\newcommand{\Q}{\mathbb{Q}}
\newcommand{\R}{\mathbb{R}}
\newcommand{\C}{\mathbb{C}}

\newtheorem{theorem}{Twierdzenie}
\newtheorem{lemma}{Lemat}
\newtheorem{defin}{Definicja}
\newtheorem{corollary}[theorem]{Wniosek}

\title{Elementarny dowód wolności grupy generowanej przez przesunięcie i
pierwszą potęgę}
\tytulang{Elementary proof of freeness of group generated by translation and
prime power}
\author{Grzegorz Milka}
\nralbumu{306320}
\kierunek{Matematyka}
\opiekun{dr. hab. Zbigniew Marciniak\\
  Faculty of Mathematics, Informatics, and Mechanics}
\dziedzina{ 
11.1 Matematyka\\ 
}
\keywords{grupa wolna, twierdzenie white'a}
\klasyfikacja{
 	12E05   	Polynomials (irreducibility, etc.)\\
 	12F99     Field extensions\\
 	20E05   	Free nonabelian groups\\
}

\begin{document}
\maketitle

\begin{abstract}
  Praca ta przedstawia elementarny dowód wolności grupy generowanej przez dwa
  przekształcenia w grupie symetrii liczb rzeczywistych: przesunięcie o liczbę
  całkowitą i podnoszenie nieparzystą, pierwszej potęgi.
\end{abstract}

\tableofcontents
\chapter{Wprowadzenie}

Teoria ciał sprawnie udowodniła niemożliwość rozwiązania odwiecznych problemów
geometrycznych takich jak kwadratura koła czy trysekcja kąta. W 1986 White
\cite{whi88} dodał kolejny naturalny przykład mocy teorii ciał. Udowodnił, że
nietrywialne przekształcenie postaci
\[
  \left( \ldots \left(\left(x + a_1\right)^{q_1} + a_2\right)^{q_2} +
    \ldots + a_n\right)^{q_n}, a_n \in Z, q_n = p^{k_i}, k_i \in \Z
\]
nie może być równe $x$.

W 1991 Cohen \cite{coh91} uogólnił wynik White'a na wymierne potęgi, gdzie
wykładniki są ułamkami nieparzystych liczb pierwszych. W 1995 Cohen i Glass
\cite{coh95} przedstawili znacznie uproszczony dowód twierdzenia White'a, który
korzysta tylko z elementarnych faktów teorii ciał i nie korzysta z teorii
Galois.

Ta praca przedstawia polski przekład dowodu Cohena i Glassa wraz z dodatkowymi
wyjaśnieniami. Twierdzenie White'a formalnie dotyczy pokazania, że grupa
generowana przez translację i potęgi jest wolna, dlatego rozdział
\ref{ch:free_groups} przedstawia podstawowe definicje i fakty dotyczące grup
wolnych. Rozdział \ref{ch:auxiliary} przedstawia dowody lematów i twierdzeń
wykorzystywanych w dowodzie głównego twierdzenia, w szczególności fundamentalny
dla twierdzenia lemat o niemożliwości zachodzenia pewnej identyczności wśród
słów wielomianowych. Dowody w tym rozdziale można pominąć przy pierwszym
czytaniu. W rozdziale \ref{ch:white_theorem} jest zaprezentowana główna teza
wraz z jej dowodem i intuicjami.

\chapter{Grupy wolne}
\label{ch:free_groups}
Podstawową ideą stojąca za pojęciem grupy wolnej $F(S)$ generowanej przez zbiór
$S$ jest to, że nie istnieje żadna nietrywialna relacja między elementami $S$ w
tej grupie. Nie istnieje żaden nietrywialny sposób uproszczenia ciągu operacji
na $S$.
Formalnie grupę wolną definiujemy następująco:

\begin{defin}[Uniwersalna własność grup wolnych]
  Niech $S$ będzie zbiorem a $F$ grupą taką, że istnieje odwzorowanie $\iota: S
  \rightarrow F$. $F$ nazywa się grupą wolną generowaną przez zbiór $S$ jeśli
  dla każdej grupy $G$ i odwzorowania $\rho: S \rightarrow F$ istnieje dokładnie
  jeden homomorfizm $\phi: F \rightarrow G$ taki, że:

  \begin{equation}
  \forall_{s \in S} \,\rho(s) = (\phi \circ \iota) (s)
    \label{eq:universal_property}
  \end{equation}
  \label{def:universal_property}
\end{defin}

Z tej własności można udowodnić, że $\iota$ jest iniekcją i obraz $\iota$
generuje $F$.

\begin{corollary}
  $\iota$ jest iniekcją.
\end{corollary}
\begin{proof}
  Jeśli $|S| < 2$ to różnowartościowość trywialnie zachodzi. W przeciwnym
  wypadku niech $a, b \in S, a \neq b$. Niech $\rho: S \rightarrow \Z_2, \rho(a)
  = 0, \rho(b) = 1$. Jeśli $\iota(a) = \iota(b)$ to $\phi(\iota(a)) =
  \phi(\iota(b))$, ale $\rho(a) \neq \rho(b)$. Sprzeczność.
\end{proof}

\begin{corollary}
  Obraz $\iota$ generuje $F$.
\end{corollary}
\begin{proof}
  Niech $G = \langle \iota(S) \rangle, \rho(s) = \iota(s)$.
  Istnieje unikalny homomorfizm $\phi: F \rightarrow \langle \iota(S) \rangle$
  spełniający \ref{eq:universal_property}.

  Ponieważ $\langle \iota(s) \rangle \subseteq F$ zatem istnieją co najmniej dwa
  możliwe homomorfizmy $\psi, \psi': F \rightarrow F$: $\psi = id$ i
  $\psi' = \phi$.

  Jeśli weźmiemy $F$ za $G$ i użyjemy uniwersalnej własności to z
  jednoznaczności $\phi$ otrzymamy, że $\psi = \psi'$, czyli $\phi = id$. Zatem
  $F = \langle \iota(S)\rangle$.
\end{proof}

\subsubsection{Dowód istnienia grupy wolnej generowanej przez zbiór $S$}
Mając dany zbiór $S$ możliwe jest skonstruowanie grupy wolnej $F(S)$ generowanej
przez ten zbiór.
Kluczowym pomysłem jest tutaj fakt, że elementy $F(S)$ muszą w pewien sposób
zakodować operacje użyte do wygenerowania podgrupy $\langle \rho(S) \rangle$,
ponieważ dla przykładu

\[
\phi\left(\iota\left(s_1\right) \ldots \iota\left(s_n\right)\right)
=
\phi\left(\iota\left(s_1\right)\right) \ldots \phi\left(\iota\left(s_n\right)\right)
=
\rho\left(s_1\right)\ldots\rho\left(s_n\right)\]

To powinno być możliwe dla każdej grupy $G$ dlatego potrzebujemy słów kodujących
elementy $s_i$.

\begin{defin}[Zbiór słów zredukowanych]
  Niech $S^{-1}$ będzie zbiorem takim, że $S \cap S^{-1} = \emptyset$ i istnieje
  bijekcja między $S$ i $S^{-1}$.
  Element zbioru $S^{-1}$ odpowiadający w tej bijekcji elementowi $s \in S$
  oznaczamy przez $s^{-1}$ i nazywamy elementem odwrotnym do $s$.
  Załóżmy dodatkowo, że $1 \not \in S \cup S^{-1}$ i $1^{-1} = 1$.
  Zbiorem słów zredukowanych $F(S)$ jest zbiór nieskończonych ciągów postaci:

  \[(s_1, s_2, \ldots)\]

  takich, że $s_i \in S \cup S^{-1} \cup \{1\}, s_i = 1$ for dostatecznie dużych
  $i$ oraz

\begin{enumerate}
  \item $s_{i+1} \neq s_i^{-1}$ dla każdego $i$ dla którego $s_i \neq 1$.
  \item Jeśli $s_k = 1$ to $\forall_{i \geq k} s_i = 1$
\end{enumerate}
\end{defin}

Słowo $(1,1,\ldots)$ nazywamy słowem pustym i będziemy oznaczać symbolem $1$.

Słowo zredukowane $
(s_1^{\epsilon_1},
s_2^{\epsilon_2}, \ldots,
s_n^{\epsilon_n}, 1, 1, \ldots)$,
gdzie $\epsilon_i \in \{-1, 1\}$ będziemy oznaczać napisem
$s_1^{\epsilon_1}\ldots s_n^{\epsilon_n}$.

\begin{defin}[Grupa wolna generowana przez zbiór]
  Grupą wolną generowaną przez zbiór $S$ nazywamy zbiór słów zredukowanych
  $F(S)$ gdzie operacją grupową $\cdot$ jest konkatenacja z eliminacją
  redukujących się symboli, tj. ciągów postaci $ss^{-1}$. Formalnie niech:
  $s = s_1^{\epsilon_1}\ldots s_n^{\epsilon_n}, p = p_1^{\epsilon_1}\ldots
  p_m^{\epsilon_n} \in F(S)$, $s_i, p_j \neq 1$ i $k \in \N$ jest takim
  największym takim indeksem, że:

  \[\forall_{l \leq k} \; s_{n - l + 1}^{\epsilon_{n - l + 1}} =
    p_l^{-\epsilon_l}\]

  Wówczas $s \cdot p = s_1^{\epsilon_1}\ldots s_{n -
  k}^{\epsilon_{n-k}}p_{k+1}^{\epsilon_{k+1}}\ldots p_{m}^{\epsilon_m}$. Jeśli
  $k$ jest równe długości któregoś ze słów to w powyższej definicji to słowo
  redukuje się do $1$.

  Iniekcję $\iota$ z definicji \ref{def:universal_property} definiujemy w
  następujący sposób: $\iota: S \rightarrow F(S), \iota(s) = s$.
  \label{def:generated_free_group}
\end{defin}

\begin{theorem}
  Grupa $F(S)$ z definicji \ref{def:generated_free_group} jest rzeczywiście grupą
  i dodatkowo jest grupą wolną generowaną przez $S$.
\end{theorem}
\begin{proof}
  Z definicji działania grupowego: $1 \cdot s = s \cdot 1 = s$ oraz
  dla każdego słowa
  $s = s_1^{\epsilon_1}\ldots s_n^{\epsilon_n}$
  jego odwrotnością jest:
  $s = s_n^{-\epsilon_n}\ldots s_1^{-\epsilon_1}$. Pozostaje tylko sprawdzić, że
  $\cdot$ jest operacją łączną. Dowód tej własności można przeprowadzić
  indukcyjnie, tutaj jednak użyję innego rozumowania.

  Zdefiniujmy $\sigma_s: F(S) \rightarrow F(S), s \in S \cup S^{-1} \cup \{1\}$,
  jako $\sigma_s(a) = \iota(s) \cdot a$. $\sigma_{s^{-1}} \circ \sigma_{s}$ jest
  identycznością zatem $\sigma_{s}$ jest permutacją na $F(S)$. Niech
  $A(S) = \left\langle \sigma_s : s \in S \cup S^{-1} \cup \{1\}\right\rangle$
  będzie podgrupą grupy symetrii $F(S)$.

  Odwzorowanie:
  \[
    s_1^{\epsilon_1} \ldots
    s_n^{\epsilon_n} \rightarrow
    \sigma_{s_1^{\epsilon_1}} \ldots
    \sigma_{s_n^{\epsilon_n}}
  \]

  jest bijekcją z $F(S)$ do $A(S)$ która zachowuje działanie grupowe obu
  zbiorów.
  Ponieważ działanie w $A(S)$ jest łączne zatem działanie $\cdot$ w
  $F(S)$ też musi być.

  Aby udowodnić wolność niech $G$ będzie dowolną grupą i $\rho: S \rightarrow G$
  dowolnym odwzorowaniem. Homomorfizm $\phi: F(S) \rightarrow G$ musi spełniać

  \[ \phi\left(s_1^{\epsilon_1} \ldots s_n^{\epsilon_n}\right) =
  \rho\left(s_1\right)^{\epsilon_1} \ldots \rho\left(s_n\right)^{\epsilon_n}\]

  To służy za definicję $\phi$ i pokazuje jej istnienie oraz jednoznaczność.
\end{proof}

\section{Unikalność grup wolnych}
Grupa słów zredukowanych jest utożsamiana z pojęciem grupy wolnej nad $S$ z
powodu następującego twierdzenia.

\begin{theorem}
  Jeśli $F, F'$ są grupami wolnymi nad $S$ to $F \cong F'$.
\end{theorem}
\begin{proof}
  Z definicji \ref{def:universal_property} istnieją jednoznaczne homomorfizmy
  $\phi: F \rightarrow F', \phi': F' \rightarrow F$, które są identycznością na
  zbiorze $S$. Zatem $\phi' \circ \phi: F \rightarrow F$ jest identycznością,
  gdyż $\iota(S)$ generuje zarówno $F$ jak i $F'$. Podobnie $\phi \circ \phi'$
  Zatem $\phi, \phi'$ są izomorfizmami.
\end{proof}

\chapter{Pomocnicze lematy}
\section{Nietrywialne pierwiastki wielomianów nie należą do $\overline{\Q}(x)$}
\begin{lemma}
  Niech $a \in \overline{\Q}, a \neq 0, f(x) \in \overline{\Q}[x], \deg f(x)
  \neq 0$ i $p \in \N$ pierwsza.  Wówczas nie istnieje taka funkcja wymierna
  $w(x) \in \overline{\Q}(x)$, że $f(x)^{p} + a = w(x)^p$. Innymi słowy
  wielomiany niebędące trywialnie $p$-tymi potęgami nie posiadają $p$-tego
  pierwiastka w ciele funkcji wymiernych.
  \label{lem:nontrivial_roots}
\end{lemma}

\begin{proof}
Załóżmy przeciwnie.

Jeśli $w(x) = \frac{m(x)}{n(x)}$, gdzie $m(x), n(x) \in \overline{\Q}[x], \deg
n(x) \neq 0$ i jest to ułamek nieskracalny, to w równaniu: $f(x)^{p} + a =
\frac{m(x)^p}{n(x)^p}$ po lewej stronie mamy wielomian. Zatem $n(x)^p | m(x)^p$,
ale założyliśmy, ze $n(x) \! \not| m(x)$. Sprzeczność.

Zatem $w(x) = g(x), g(x) \in \overline{\Q}[x]$. Mamy:
\begin{eqnarray*}
  f(x)^p + a &=& g(x)^p \\
  a &=& g(x)^p - f(x)^p\\
  a &=& \left(g(x) - f(x)\right)\left(\sum_{i=0}^{p-1}g(x)^if(x)^{p-1 - i}\right)\\
\end{eqnarray*}

Możemy założyć, że $f(x)$ jest wielomianem monicznym. Z końcowego równania $g(x)
- f(x)$ dzieli $a$, zatem $g, f$ są tego samego stopnia. Zatem w prawym
czynniku mamy wielomian stopnia $(p-1)\deg g(x)$ co jest większe od zera.
Sprzeczność.
\end{proof}

\section{Lematy o słowach wielomianowych} 
\begin{lemma}
  Każdy pierwiastek słowa wielomianowego $f$ jest pierwiastkiem o krotności
  będącej potęgą $p$.
\end{lemma}

\begin{proof}
Definiujemy ciąg słów wielomianowych $f_1, f_2, \ldots, f_{l+1}$ jako ciąg
kolejnych aplikacji funkcji postaci $(x+a_i)^{p^{k_i}}$ z postaci
\ref{eq:gp_element}. Tj. $f_1(x) = \left(x + a_1\right)^{p^{k_1}},
f_2(x) = \left(f_1(x) + a_2\right)^{p^{k_2}}, \ldots, f_{l+1} = f$.

Różniczkując otrzymujemy:
\[f' = \prod_{i=1}^l p^{k_i} f_i^{p^{k_i - 1}}\]

Zatem każdy wielokrotny pierwiastek $f$ jest także pierwiastkiem co najmniej
jednego z wielomianów $f_j$. Niech $f_{j_1}, \ldots, f_{j_s}$, gdzie $1 \leq j_1
\leq \ldots \leq j_s \leq l$ będą wszystkimi wyrazami z ciągu, dla których dany
pierwiastek $f$ jest również i ich pierwiastkiem. Wówczas stopień tego
pierwiastka to $\prod_{i=1}^s p^{k_{i_s}}$.
\end{proof}

\begin{lemma}
  Niech $f$ będzie słowem wielomianowym ($a_1, \ldots, a_{l+1} \neq 0$) i
  $\omega$ ($\neq 1$) będzie $p$-tym pierwiastkiem jedności w $\overline{\Q}$.
  Wówczas identyczność postaci
  \begin{equation}
    f(x)f^t(\omega x)g\left(x^p\right) = h^p(x)
    \label{eq:pol_lemma}
  \end{equation}

  gdzie $t \in \Z$ a $g, h \in \overline{Q}[x]$ jest niemożliwa.
  \label{lem:main_polynomial_lemma}
\end{lemma}

\begin{proof}
  [TODO Czy w ciele overlineQ pierwiastki p-tego stopnia są różne?]

  Jeśli \ref{eq:pol_lemma} zachodzi to możemy założyć, że $0 \leq t \leq p - 1$
  poprzez włączenie $f^p(\omega x)$ do $h^p(x)$. 

  [TODO nie rozumiem jak] Dodatkowo z \ref{eq:pol_lemma} mamy
  \begin{equation}
    f(x)f^t(\omega x) = g\left(x^p\right)h^p(x)
    \label{eq:pol_lemma_right}
  \end{equation}

  Niech $\alpha$ będzie jednokrotnym pierwiastkiem $f$. Wówczas każdy
  pierwiastek $\beta$ wielomianu $\left(x + a_1\right)^{q_1} - \left(\alpha +
  a_1\right)^{q_1}$ jest również jednokrotnym pierwiastkiem $f$. Tak jest, bo
  wówczas $f_1(\beta) = f_1(\alpha)$, a $f_1(\alpha)$ jest jednokrotnym
  pierwiastkiem $f_2f_3 \ldots f_{l}$.

  Niech, więc $f^*(x)$ będzie iloczynem czynników $(x-\alpha)$, gdzie $\alpha$
  jest jednokrotnym pierwiastkiem $f$. Wówczas
\end{proof}










\chapter{Twierdzenie White'a}

\section{Wstęp}

Niech $p$ będzie dowolną nieparzystą liczbą pierwszą. Rozważmy permutacje $t :
x \rightarrow  x+1$ i $e: x \rightarrow x^p$ zbioru liczb rzeczywistych. W 1986
Samuel White \cite{whi88} udowodnił, że dla dowolnych liczb liczb całkowitych
$a_1, \ldots, a_n, k_1, \ldots, k_n$, gdzie $a_2, \ldots, a_n, k_1, \ldots,
k_{n-1}$ są niezerowe, odwzorowanie 

\begin{equation}
  t^{a_1}e^{k_1}\ldots t^{a_n}e^{k_n}
\label{eq:gp_element}
\end{equation}

nie jest identycznością.  Innymi słowy, podgrupa $G_p$ grupy $\Sigma_\R$
generowana przez $t, e$ jest wolna.

Słowa zredukowane postaci \ref{eq:gp_element}, gdzie wykładniki $k_i$ są
nieujemne, będziemy nazywać \textbf{słowami wielomianowymi}.

W tym rozdziale przedstawimy dowód tego twierdzenia oraz faktów pomocniczych.

\begin{defin}[Rozszerzenie czysto przestępne]
  Rozszerzenie $L$ ciała $K$ nazywamy czysto przestępnym wtw., gdy istnieje
  zbiór algebraicznie niezależny $A \subseteq L$ taki, że $K(A) = L$. Równoważne
  gdy $L$ jest izomorficzne z ciałem funkcji wymiernych w ciele $K$ nad
  zmiennymi $\left\{X_a : a \in A\right\}$.
\end{defin}

\section{Przestępny łańcuch}
\label{sec:przelan}

Niech $K = \bar{\Q}$ - domknięcie ciała liczb wymiernych, czyli zbiór wszystkich
liczb algebraicznych.

Niech $\xi$ to będzie dowolna liczba przestępna. Potrzebujemy zdefiniować ciąg
liczb przestępnych będących wynikiem kolejnych aplikacji liter tworzących słowo
$w$ na $\xi$. W tym celu oznaczmy sobie możliwe elementy. Niech $e_m: x
\rightarrow x^{p^m}, m \geq 1$ a jej odwrotność to $r_m: x \rightarrow
x^{\frac{1}{p^m}}, m \geq 1$. Translację oznaczymy jako $t_a: x \rightarrow x +
a, a \in \N, a \neq 0$.  Wówczas słowo $w$ to ciąg symboli postaci $w =
v_1\ldots v_n$, gdzie $v_i$ to $t_a$, $e_m$, lub $r_m$. 

Definiujemy $n+1$-elementowy ciąg $\left(\xi_i\right)$ jako: $\xi_1 = \xi,
\xi_{j+1} = \xi_{j}v_j, j \in \{1, \ldots, n\}$. W szczególności $\xi_{n+1} =
w(\xi)$.

Rozważmy teraz ciąg rozszerzeń ciała $K$:
$
K\left(\xi_1\right),
K\left(\xi_1, \xi_2\right),
K\left(\xi_1, \xi_3\right),
\ldots,
K\left(\xi_1, \xi_{n+1}\right)$.

Twierdzę, że ten ciąg tworzy wieżę:

\begin{equation}
K\left(\xi_1\right) \subseteq
K\left(\xi_1, \xi_2\right) \subseteq
K\left(\xi_1, \xi_3\right) \subseteq 
\ldots
\subseteq
K\left(\xi_1, \xi_{n+1}\right)
\label{eq:tower}
\end{equation}

Załóżmy, że \ref{eq:tower} zostało udowodnione. Z tego ciągu wynika nierówność
$w\left(\xi\right) \neq \xi$. Aby to zauważyć wystarczy rozważyć dwa przypadki.
Jeśli wśród $v_i$ nie ma pierwiastków to $w$ jest wielomianem nad $\Q$. Wówczas
$w(\xi) = \xi$ jest sprzeczne z przestępnością $\xi$ zatem $w(\xi) \neq \xi$.
Niech, więc $v_j = r_m$ będzie pierwszym pierwiastkiem wśród napisu $w$. Wówczas
$\xi_j \in K\left(\xi_1\right)$ i pokażę również, że $\xi_{j}^{\frac{1}{p}} \not
\in K(\xi_1)$. Zatem $K\left(\xi_1\right) \subset K\left(\xi_1,
\xi_{j+1}\right)$. Z \ref{eq:tower} mamy zatem, że $K\left(\xi_1\right) \neq
  K\left(\xi_1, \xi_{n+1}\right)$, czyli $w(\xi) \neq \xi$.

Dowód \ref{eq:tower} zostanie przeprowadzony indukcyjne, równocześnie z dwoma
innymi własnościami.

\section{Podłańcuch}

Zamiast rozpatrywać $w$ jako napis symboli $v_i$, zbierzemy je w sylaby: $s_1,
\ldots, s_k$, gdzie każda sylaba kończy się pierwiastkiem $r_{m_j}$. Niech słowo
$f \in G_p$ będzie słowiem wielomianowym, jeśli $f$ składa się tylko z
translacji $t_a$ i dodatnich potęg $e_m$, czyli $f(x)$ jest po prostu
wielomianem. Możemy wówczas napisać, że $w = s_1\ldots s_k$, gdzie $s_j =
f_jr_{m_j}$, $f_j$ jest $j$-tym słowem wielomianowym. Aby zachować proste
oznaczenia pozwalamy na to by $f_1 = x$  i $s_k = f_k$, w tym przypadku będziemy
przyjmowali, że $m_k = 0$. 

Podobnie jak w 
\ref{sec:przelan} definiujemy podłańcuch $\left\{\mu_1, \ldots,
\mu_{k+1}\right\}$ jako $\mu_1 = \xi_1 = \xi, \mu_{j+1} = s_j\left(\mu_j\right)$
dla $1 \leq j \leq k$. Wówczas $\mu_{k+1} = \xi_{n+1} = w(\xi)$.

Zauważmy, że jeśli zachodzi \ref{eq:tower} to zachodzi też następujące:

\[
  K\left(\mu_1\right) \subseteq
  K\left(\mu_1, \mu_2\right) \subseteq
  \ldots \subset
  K\left(\mu_1, \mu_{k+1}\right)
\]

Pokazaliśmy już, że pierwsze zawieranie jest ścisłe. Drugą rzeczą, którą
będziemy dowodzić jest właśnie:

\begin{equation}
  K(\mu_1) \subset
  K(\mu_1, \mu_2) \subset
  \ldots \subset
  K(\mu_1, \mu_{k+1})
  \label{eq:strict_tower}
\end{equation}

Jeśli $m_k = 0$ to ostatnie zawieranie nie jest ścisłe.

[TODO czy zawieranie ścisłe to właściwy zwrot]

Do trzeciej własności zauważmy, że $\mu_2^{p^{m_1}} = f_1\left(\mu_1\right)$.
Zatem $K\left(\mu_1\right)\left(\sqrt[p^{m_1 -
1}]{f_1\left(\mu_1\right)}\right)$ jest rozszerzeniem prostym
$K\left(\mu_1\right)$ o stopniu $p$.
[TODO w pracy tutaj jest pierwiastek stopnia p i rozszerzenia stopnia p. Czy to
jest błąd?]
To rozszerzenie jest zawarte w $K\left(\mu_1, \mu_{k+1}\right)$. Czy może
istnieć inne rozszerzenie proste o tej własności? Okazuje się, że nie.

\section{Rezultat}

Będziemy dowodzić 3 fakty równoczesną indukcyjną po długości $n$ słowa $w$.

Oznaczmy $K\left(\mu_1, \mu_j\right)$ jako $K_j$.

\begin{theorem}
  Niech $w = v_1 \ldots v_n = s_1 \ldots s_k$ będzie słowem w $G_p$ i $\xi$
  rzeczywistą liczbą przestępną. Wówczas

  \begin{description}
    \item{$H_1$:} 
      $K\left(\xi_1\right)\subseteq
      K\left(\xi_1, \xi_2\right)\subseteq
      \ldots \subseteq
      K\left(\xi_1, \xi_{n+1}\right)$
    \item{$H_2$:}
      $K_1 \subset K_2 \subset \ldots \subset K_{k+1}$ (ostatnie zawieranie jest
      równością, jeśli $m_k = 0$)
    \item{$H_3$:}
      Jeśli $F$ jest prostym rozszerzeniem $K_1$ rzędu $p$  zawartym w $K_{k+1}$
      to $F \subseteq K_2$.
  \end{description}
  \label{th:hypothesis_h}
\end{theorem}

Pokazaliśmy już jak z \ref{th:hypothesis_h} wynika

\begin{corollary}[Twierdzenia White'a \cite{whi88}]
  Grupa $G_p$ jest grupą wolną rzędu 2.
\end{corollary}

Niech $H(n)$ oznacza prawdziwość Hipotezy H dla słów długości nieprzekraczającej
$n$.

Dla $n=1$ mamy:
\begin{description}
  \item{$H_1$:} 
    $K\left(\xi_1\right) \subseteq K\left(\xi_1, \xi_2\right)$, co jest
    oczywiste.
  \item{$H_2$:} Dla $m_1 = 0$ ta hipoteza nic nie stwierdza. W przeciwnym
    wypadku ta hipoteza stwierdza, że $K\left(\xi_1\right) \subset K\left(\xi_1,
    \xi_2\right)$. $\xi_2$ jest pierwiastkiem $\xi_1$ o stopniu podzielnym przez
    $p$, więc wykorzystując lemat \ref{lem:nontrivial_roots} i znany fakt
    $K\left(\xi_1\right) \cong K[x]$ mamy, że $\xi_2 \not \in
    K\left(\xi_1\right)$.
  \item{$H_3$:} Prawdziwe, bo $K_{k+1} = K_2$.

\end{description}

Zatem będziemy przyjmować, że $n \geq 2$. Ciąg wywodu będzie formować wg.
następującego schematu:

\[ H(n) \Rightarrow H_1(n+1) \Rightarrow H_2(n+1) \Rightarrow H_3(n+1)
\Rightarrow H(n+1)\]

Zauważmy, że jeśli słowo $w$ zaczyna się lub kończy przesunięciem to krok
indukcyjny jest łatwy do udowodnienia. Zaiste, jeśli $w$ kończy się translacją
to $H_2(n+1)$ trywialnie wynika z $H(n)$.  Z równości $K\left(\xi_1,
\xi_{n}\right) = K\left(\xi_1, \xi_{n} + a\right), \xi_{n+1} = \xi_n + a$ wynika
$H_1(n+1), H_3(n+1)$. Jeśli $w$ zaczyna się translacją to podobnie tylko tym
razem korzystamy z równości $K\left(\xi_1, \xi_1 + a\right) =
K\left(\xi_1\right)$. Zatem będziemy zakładać, że $w$ zaczyna i kończy się $e_m$
lub $r_m$ dla pewnego $m \geq 1$.

\section{Dowód $H_1(n+1)$}
Przyjmujemy notację i założenia z poprzednich sekcji oraz hipotezę $H(n)$.
Będziemy też używać notacji typu $\left(\xi_i, \xi_j\right)$ do oznaczenia
podsłowa $v_i\ldots w_{j-1}$ słowa $w$ którego działanie przekształca $\xi_i$ na
$\xi_j$. Dokonamy teraz dodatkowe obserwacje i poprawki.

Z $H_1(n)$ zaaplikowanego do $\left(\xi_1, \xi_{n+1}\right)$ oraz $\left(\xi_2,
\xi_{n+2}\right)$ mamy

\begin{equation}
  K\left(\xi_1\right) \subseteq
  K\left(\xi_1, \xi_2\right) \subseteq
  \ldots \subseteq
  K\left(\xi_1, \xi_{n+1}\right)
  \label{eq:h_1_first}
\end{equation}

\begin{equation}
  K\left(\xi_2\right) \subseteq
  K\left(\xi_2, \xi_3\right) \subseteq
  \ldots \subseteq
  K\left(\xi_2, \xi_{n+2}\right)
  \label{eq:h_1_second}
\end{equation}

Załóżmy, że $H_1(n+1)$ nie zachodzi, tj. $K\left(\xi_1, \xi_{n+2}\right)$ nie
zawiera $K\left(\xi_1, \xi_{n+1}\right)$. Wówczas $v_{n+1} = e_m$ i $m_k = 0$.
  $\xi_{n+2} = v_{n+1}\left(\xi_{n+1}\right) \in K\left(\xi_{n+1}\right)$, czyli
  $K\left(\xi_1, \xi_{n+2}\right) \subset K\left(\xi_1, \xi_{n+1}\right)$.

\begin{verbatim}
[TODO Napisać, że jeśli xi_2 zawiera sie w K(1, n+2) to H_1(n+1) zachodzi]

[TODO Autor tam dopasowuje \xi_1 i v_1 tak by czesc wspolna K(1, n+2), K_2 byla
  rowna K_1. Z poprzedniego wynika, że tak sie da zrobic, bo czesc wspolna nie
  może być równa K_2. Sprawdzić, czemu autor tak robi, tj. czy jest to konieczne
  i czemu to nie psuje dowodu]
  \end{verbatim}

\begin{thebibliography}{99}
\addcontentsline{toc}{chapter}{Bibliografia}

\bibitem[Bea65]{beaman} Juliusz Beaman, \textit{Morbidity of the Jolly
    function}, Mathematica Absurdica, 117 (1965) 338--9.
\end{thebibliography}

\end{document}
