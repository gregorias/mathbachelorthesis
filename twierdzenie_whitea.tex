\chapter{Twierdzenie White'a}
\label{ch:white_theorem}
\section{Wstęp}

Niech $p$ będzie dowolną nieparzystą liczbą pierwszą. Rozważmy permutacje 
${t : x \rightarrow  x+1}$ i~${e: x \rightarrow x^p}$ zbioru liczb
rzeczywistych. W 1986 Samuel White \cite{whi88} udowodnił, że dla dowolnych
liczb liczb całkowitych
$a_1, \ldots, a_n, k_1, \ldots, k_n$, gdzie $a_2, \ldots, a_n, k_1, \ldots,
k_{n-1}$ są niezerowe, odwzorowanie
\begin{equation*}
  t^{a_1}e^{k_1}\ldots t^{a_n}e^{k_n}
\end{equation*}
nie jest identycznością. To jest istnieje taka liczba rzeczywista $x$, że dla
$q_i$ postaci $p^{k_i}$:
\begin{equation*}
  \left( \ldots \left(\left(x + a_1\right)^{q_1} + a_2\right)^{q_2} +
    \ldots + a_n\right)^{q_n}
\end{equation*}
nie jest równe $x$. Innymi słowy podgrupa $G_p = \langle t, e \rangle$ grupy
$\Sigma_\R$ generowana przez $t, e$ jest wolna.

W tym rozdziale przedstawimy dowód tego twierdzenia oraz faktów pomocniczych.

\section{Przygotowania}
\label{sec:preparation}
Niech $K = \overline{\Q}$ - domknięcie ciała liczb wymiernych, czyli zbiór
wszystkich liczb algebraicznych.
Niech $\xi$ to będzie dowolną liczba przestępna.
Potrzebujemy zdefiniować ciąg liczb przestępnych będących wynikiem kolejnych
aplikacji liter tworzących słowo $w$ na $\xi$.
W tym celu oznaczmy sobie możliwe elementy słowa $w$.
Niech $e_m: x \rightarrow x^{p^m}, m \geq 1$ a jej odwrotność to $r_m: x
\rightarrow x^{\frac{1}{p^m}}, m \geq 1$.
Translację oznaczymy jako $t_a: x \rightarrow x + a, a \in \N, a \neq 0$.
Słowo $w$~będziemy czasem traktowali jako iloczyn postaci $w =
v_1\ldots v_n$, gdzie $v_i$ to $t_a$, $e_m$ lub $r_m$.

Możemy teraz zdefiniować $n+1$-elementowy ciąg $\left(\xi_i\right)$ jako:
\[\xi_1 = \xi, \xi_{j+1} = v_j\left(\xi_j\right), j \in \{1, \ldots, n\}\]
W szczególności $\xi_{n+1} = w(\xi)$.

Oprócz rozważania pojedynczych generatorów w słowie $w$, konieczne jest
rozpatrzenie całych podsłów kończących się pierwiastkiem. Połączmy podsłowa
$v_i$ w sylaby $s_1, \ldots, s_k$, gdzie każda sylaba jest postaci:
$f_jr_{m_j}$, $f_j$ jest słowem wielomianowym a $m_j \geq 1$ dla
$1 \leq j \leq k-1$. Zatem $w = \prod_{i=1}^k s_i$.
Aby zachować prostotę oznaczenia pozwalamy na to by $f_1 = x$  i $s_k = f_k$, w
tym przypadku będziemy przyjmowali, że $m_k = 0$.

Podobnie jak w dla $\xi_i$ definiujemy ciąg $\left(\mu_i\right)$ jako
\[\mu_1 = \xi_1 = \xi, \mu_{j+1} = s_j\left(\mu_j\right)\]
dla $1 \leq j \leq k$. Wówczas $\mu_{k+1} = \xi_{n+1} = w(\xi)$.

Rozważmy teraz ciąg rozszerzeń ciała $K$:
$
K\left(\xi_1\right),
K\left(\xi_1, \xi_2\right),
K\left(\xi_1, \xi_3\right),
\ldots,
K\left(\xi_1, \xi_{n+1}\right)$ i
$
K\left(\mu_1\right),
K\left(\mu_1, \mu_2\right),
\ldots,
K\left(\mu_1, \mu_{n+1}\right)$.

Aby udowodnić twierdzenie White'a potrzebne będzie indukcyjne udowodnienie
silniejszej tezy. Mianowicie, że te dwa ciągi rozszerzeń tworzą wieże:

\begin{equation}
K\left(\xi_1\right) \subseteq
K\left(\xi_1, \xi_2\right) \subseteq
K\left(\xi_1, \xi_3\right) \subseteq
\ldots
\subseteq
K\left(\xi_1, \xi_{n+1}\right)
\label{eq:tower}
\end{equation}

\begin{equation}
  K(\mu_1) \subsetneq
  K(\mu_1, \mu_2) \subsetneq
  \ldots \subsetneq
  K(\mu_1, \mu_{k+1})
  \label{eq:strict_tower}
\end{equation}
Jeśli $s_k$ nie kończy się pierwiastkiem to ostatnie zawieranie właściwe nie
zachodzi.

Załóżmy, że \ref{eq:tower} zostało udowodnione. Z tej wieży wynika nierówność
$w\left(\xi\right) \neq \xi$. Aby to zauważyć wystarczy rozważyć dwa przypadki.
Jeśli wśród $v_i$ nie ma pierwiastków to $w$ jest wielomianem nad $\Q$. Wówczas
$w(\xi) = \xi$ jest sprzeczne z przestępnością $\xi$ zatem $w(\xi) \neq \xi$.
Niech, więc $v_j = r_m$ będzie pierwszym pierwiastkiem wśród napisu $w$. Wówczas
$\xi_j \in K\left(\xi_1\right)$. Z lematu \ref{lem:nontrivial_roots} mamy, że
$\xi_{j}^{\frac{1}{p}} \not \in K(\xi_1)$ (dodatkowo \ref{eq:strict_tower} to
stwierdza). Zatem $K\left(\xi_1\right) \subsetneq K\left(\xi_1,
\xi_{j+1}\right)$. Z \ref{eq:tower} mamy zatem, że $K\left(\xi_1\right) \neq
K\left(\xi_1, \xi_{n+1}\right)$, czyli $w(\xi) \neq \xi$.

Dowód obu własności zostanie przeprowadzony indukcyjne. Dodatkowo będziemy
potrzebować jeszcze trzeciej własności, którą będziemy dowodzić równocześnie z
obiema powyższymi. Zauważmy, że $\mu_2^{p^{m_1}} = f_1\left(\mu_1\right)$.
Zatem $K\left(\mu_1\right)\left(\sqrt[p]{f_1\left(\mu_1\right)}\right)$ jest
rozszerzeniem $K\left(\mu_1\right)$ o stopniu $p$.
To rozszerzenie jest zawarte w $K\left(\mu_1, \mu_2\right)$ i
$K\left(\mu_1, \mu_{k+1}\right)$.
Czy może istnieć inne rozszerzenie stopnia $p$, które zawiera się w
$K\left(\mu_1, \mu_{k+1}\right)$, ale nie zawiera się w $K\left(\mu_1,
\mu_2\right)$? Okazuje się, że nie.

\section{Rezultat}

Będziemy dowodzić 3 fakty równoczesną indukcyjną po długości $n$ słowa $w$.

Oznaczmy $K\left(\mu_1, \mu_j\right)$ jako $K_j$.

\begin{theorem}
  Niech $w = v_1 \ldots v_n = s_1 \ldots s_k$ będzie słowem w $G_p$ i $\xi$
  rzeczywistą liczbą przestępną. Wówczas

  \begin{description}
    \item{$H_1$:}
      $K\left(\xi_1\right)\subseteq
      K\left(\xi_1, \xi_2\right)\subseteq
      \ldots \subseteq
      K\left(\xi_1, \xi_{n+1}\right)$
    \item{$H_2$:}
      $K_1 \subsetneq K_2 \subsetneq \ldots \subsetneq K_{k+1}$ (ostatnie zawieranie jest
      równością, jeśli $m_k = 0$)
    \item{$H_3$:}
      Jeśli $F$ jest rozszerzeniem $K_1$ rzędu $p$ zawartym w $K_{k+1}$
      to $F \subseteq K_2$.
  \end{description}
  \label{th:hypothesis_h}
\end{theorem}

\begin{corollary}
  Rozszerzenie $K_{k+1}/K_1$ jest stopnia $p^m, m = \sum_{i=1}^k m_i$.
\end{corollary}

\begin{proof}
  Wystarczy pokazać, że $\left[K_{j + 1} : K_j\right]
  = m_j, 1 \leq j \leq k$. Reszta wynika z $H_2$.  Stopień pojedynczego
  rozszerzenia wynika wprost z lematu \ref{lem:root_extension}.
\end{proof}

Pokazaliśmy już jak z \ref{th:hypothesis_h} wynika

\begin{corollary}[Twierdzenia White'a \cite{whi88}]
  Grupa $G_p$ jest grupą wolną rzędu 2.
\end{corollary}

Niech $H(n)$ oznacza prawdziwość Hipotezy H dla słów długości nieprzekraczającej
$n$.

Dla $n=1$ mamy:
\begin{description}
  \item{$H_1$:}
    $H_1(1)$ stwierdza, że
    $K\left(\xi_1\right) \subseteq K\left(\xi_1, \xi_2\right)$, co jest
    oczywiste.
  \item{$H_2$:} Dla $m_1 = 0$ ta hipoteza nic nie stwierdza. W przeciwnym
    wypadku ta hipoteza stwierdza, że $K\left(\xi_1\right) \subsetneq K\left(\xi_1,
    \xi_2\right)$. $\xi_2$ jest pierwiastkiem $\xi_1$ o stopniu podzielnym przez
    $p$, więc wykorzystując lemat \ref{lem:nontrivial_roots} i znany fakt
    $K\left(\xi_1\right) \cong K(x)$ mamy, że $\xi_2 \not \in
    K\left(\xi_1\right)$.
  \item{$H_3$:} Prawdziwe, bo $K_{k+1} = K_2$.
\end{description}

Zatem będziemy przyjmować, że $n \geq 2$. Ciąg wywodu będzie formować wg.
następującego schematu:

\[ H(n) \Rightarrow H_1(n+1) \Rightarrow H_2(n+1) \Rightarrow H_3(n+1)
\Rightarrow H(n+1)\]

Zauważmy, że jeśli słowo $w$ zaczyna się lub kończy przesunięciem to krok
indukcyjny jest łatwy do udowodnienia. Zaiste, jeśli $w$ kończy się translacją
to $H_2(n+1)$ trywialnie wynika z $H(n)$.  Z równości $K\left(\xi_1,
\xi_{n}\right) = K\left(\xi_1, \xi_{n} + a\right), \xi_{n+1} = \xi_n + a$ wynika
$H_1(n+1), H_3(n+1)$. Jeśli $w$ zaczyna się translacją to podobnie tylko tym
razem korzystamy z równości $K\left(\xi_1, \xi_1 + a\right) =
K\left(\xi_1\right)$. Zatem będziemy zakładać, że $w$ zaczyna i kończy się $e_m$
lub $r_m$ dla pewnego $m \geq 1$.

\section{Dowód $H_1(n+1)$}
Przyjmujemy notację i założenia z poprzednich sekcji oraz hipotezę $H(n)$.
Będziemy też używać notacji typu $\left(\xi_i, \xi_j\right)$ do oznaczenia
podsłowa $v_i\ldots v_{j-1}$ słowa $w$ którego działanie przekształca $\xi_i$ na
$\xi_j$. Dokonamy teraz dodatkowych obserwacji i poprawek.

Z $H_1(n)$ zaaplikowanego do $\left(\xi_1, \xi_{n+1}\right)$ oraz $\left(\xi_2,
\xi_{n+2}\right)$ mamy

\begin{equation}
  K\left(\xi_1\right) \subseteq
  K\left(\xi_1, \xi_2\right) \subseteq
  \ldots \subseteq
  K\left(\xi_1, \xi_{n+1}\right)
  \label{eq:h_1_first}
\end{equation}

\begin{equation}
  K\left(\xi_2\right) \subseteq
  K\left(\xi_2, \xi_3\right) \subseteq
  \ldots \subseteq
  K\left(\xi_2, \xi_{n+2}\right)
  \label{eq:h_1_second}
\end{equation}

Załóżmy, że $H_1(n+1)$ nie zachodzi, tj. $K\left(\xi_1, \xi_{n+1}\right)$ nie
zawiera się w $K\left(\xi_1, \xi_{n+2}\right)$. Wówczas $v_{n+1} = e_m$ i $m_k =
0$.
Naturalnie $\xi_{n+2} = v_{n+1}\left(\xi_{n+1}\right) \in
K\left(\xi_{n+1}\right)$, czyli 
${K\left(\xi_1, \xi_{n+2}\right) \subsetneq
K\left(\xi_1, \xi_{n+1}\right)}$.
Zauważmy, że jeśli $\xi_2 \in K\left(\xi_1\right)$ to z ostatniego zawierania w
\ref{eq:h_1_second} mamy:
\[
  K\left(\xi_1, \xi_{n+1}\right) =
  K\left(\xi_2, \xi_{n+1}\right)\left(\xi_1\right) \subseteq
  K\left(\xi_2, \xi_{n+2}\right)\left(\xi_1\right) =
K\left(\xi_1, \xi_{n+2}\right)\]
Zatem musimy założyć, że $v_1 = r_{m_1}$ aby mieć $\xi_2 \not \in
K\left(\xi_1\right)$.

W celu uniknięcia wprowadzania dodatkowych symboli będziemy chcieli, aby
zachodziło 
\begin{equation}
K\left(\xi_1, \xi_{n+2}\right) \cap
K\left(\xi_1, \xi_2\right) = K\left(\xi_1\right)
\label{eq:h_1_start_simplification}
\end{equation}
W tym celu podstawiamy $\xi_1^* = \xi_2^{p^{m_1-1}}$ i $v_1^* = r_1$. To
podstawienie daje nam \ref{eq:h_1_start_simplification}, gdyż
$\left[K\left(\xi_1^*, \xi_2\right):K\left(\xi_1^*\right)\right] = p$.
To podstawienie nie osłabia nam dowodzonej tezy $H_1(n+1)$, gdyż
$K\left(\xi_1\right) \subseteq K\left(\xi_1^*\right)$. Od teraz traktujemy
$\xi_1$ jako $\xi_1^*$ i $v_1$ jako $v_1^*$.

Zauważmy, że
$\xi_{n+2} \in K\left(\xi_{n+1}\right)$ i $\xi_{1} \in K\left(\xi_{2}\right)$.
Zatem z \ref{eq:h_1_first} i \ref{eq:h_1_second}
\[
  K\left(\xi_1, \xi_{n+1}\right) =
  K\left(\xi_2, \xi_{n+2}\right) =
  K\left(\xi_2, \xi_{n+1}\right)
\]

Używając sylab powyższe daje
\[
  K\left(\mu_1, \mu_{k}\right) =
  K\left(\mu_2, \mu_{k+1}\right) =
  K\left(\mu_2, \mu_{k}\right)
\]
Jest to ciało, które właściwie zawiera $K_{k+1}$. Dodatkowo $K_{k+1} \cap K_2 =
K_1$ i $K_k = K_{k+1}\left(\mu_2\right)$ jest pojedynczym rozszerzeniem
$K_{k+1}$.
$K_j = K\left(\mu_2, \mu_j\right)$ dla $2 \leq j \leq k$, gdyż $\mu_1 \in
K\left(\mu_2\right)$ i $\mu_2 \in K_j$.
Z tego, że $\mu_1 \in K_k, \mu_2 = \sqrt[p]{\mu_1}$ możemy zauważyć, że
$\left[K_k : K_{k+1}\right] = p$.

Będziemy chcieli pokazać, że $k$ musi być małe. Załóżmy , że $k \geq 3$.

Niech $\omega \neq 1$ będzie $p$-tym pierwiastkiem jedności zawartym w $K_k$.
Możemy zdefiniować $K_{k+1}$-automorfizm $\tau$ ciała $K_k$, który przekształca
$\mu_2$ na $\omega \mu_2$.
Zauważmy, że 
\[K_2 = K\left(\mu_2\right) \cong K\left(\omega \mu_2\right) =
\overline{K_2}\]
Niech $\overline{\mu_3} = \tau(\mu_3) \in K_k$ i niech druga sylaba, $s_2$,
słowa $w$ będzie postaci $fr_l$.
Działanie $\tau$ na $\mu_3^{p^l} = f\left(\mu_2\right)$ daje
$\overline{\mu_3}^{p^l} = f\left(\omega\mu_2\right)$. Zatem $K_3$ i
$\overline{K_3} = K\left(\mu_2, \overline{\mu_3}\right)$ są prostymi
rozszerzeniami stopnia $p^l$ ciała $K_2$ zawartego w $K_k$.

Jeśli $l = 1$ to użycie $H_3(n)$ do podsłowa $\left(\mu_2, \mu_k\right)$ daje
równość $\overline{K_3} = K_3$. Jeśli $l > 1$ to zauważmy, że jeśli $K_3 \cap
\overline{K_3} = K_2$ to $K_2\left(\overline{\mu_3}^{p^{l-1}}\right)$ jest
prostym rozszerzeniem $K\left(\mu_2\right)$ stopnia $p$ zawartym w $K_k$, ale
nie w $K_3$. Sprzeczność. Jeśli $K_2 \subsetneq K_3 \cap \overline{K_3}$ to
wystarczy odpowiednio dostosować $\mu_2$ i $l$ aby otrzymać równość
$K_2 = K_3 \cap \overline{K_3}$.

Teraz z lematu \ref{lem:associated_extensions} równość $K_2\left(\mu_3\right) =
K_2\left(\overline{\mu_3}\right)$ implikuje, że dla pewnego $t$ 
($p \!\! \not \! | \; t$)
mamy $\mu_3\overline{\mu_3}^t \in K\left(\mu_2\right)$. Podnosząc to do
$p^l$-tej potęgi i zastępując $\mu_2$ przez $x$ otrzymamy równanie
\[ f(x)f^t(\omega x) = h^{p^l}(x)\]
To jest niemożliwe z lematu \ref{lem:main_polynomial_lemma}. 

Zatem $k = 2$ i $w = r_{m_1}f$, gdzie $f$ jest słowem wielomianowym postaci
$f_0^{p^l}(x)$, gdzie $v_{n+1} = e_l$. Mamy zatem $K_3 = K_1$, gdyż 
$\left[K_2 : K_3\right] = \left[K_2 : K_1\right] = p$ i $K_1 \subseteq K_3$.
W takim razie $\mu_3 = f\left(\mu_2\right) \in K_1 = K\left(\mu_2^p\right)$. To
prowadzi do $f(x) = g\left(x^p\right)$ i sprzeczności.

\section{Dowód $H_2(n+1)$}
Zakładamy $H_1(n+1), H(n)$. Będziemy używać założenie indukcyjne na odwrotności
podsłowa $w$.

Mając $H_2(n)$ musimy udowodnić, że $K_k \subsetneq K_{k+1}$, gdy $m_k \geq 1$.
Załóżmy, że $H_2(n+1)$ nie zachodzi, czyli $K_k = K_{k+1}$. To oznacza, że
$\mu_{k+1} \in K_k$, zatem naturalnie $\mu_{k+1}^{p^{m_k -1}} \in K_k$. Możemy
więc zastąpić $v_{n + 1} = r_{m_k}$ przez $r_1$ a przez to poprzednie
$\mu_{k+1}$ staje się równe $\mu_{k+1}^{p^{m_k - 1}}$.
Jeśli $\mu_{k+1} \in K\left(\xi_2, \mu_k\right)$ to $K\left(\xi_2, \mu_k\right)
= K\left(\xi_2, \mu_{k+1}\right)$ co zaprzecza $H_2(n)$ zaaplikowanej do
$\left(\xi_2, \xi_{n+2}\right)$. Zatem $\mu_{k+1} \not \in K\left(\xi_2,
\mu_k\right)$ i $w$ musi zaczynać się od potęgi $v_1 = e_q$.

Z hipotezy $H_1(n+1)$ mamy
\[
  K\left(\xi_2, \mu_k\right) \left(\mu_{k+1}\right)
  =
  K\left(\xi_2, \mu_{k+1}\right)
  \subseteq
  K\left(\xi_1, \mu_{k+1}\right)
  =
  K\left(\xi_1, \mu_{k}\right)
  =
  K\left(\xi_2, \mu_{k}\right)\left(\xi_1\right)
\]

Poprzez tą samą metodę co w poprzedniej sekcji dostosowujemy tak $\xi_1$ aby móc
założyć $q = 1$. Wówczas dalej będzie $\xi_1 \not \in K\left(\xi_2,
\mu_k\right)$, ale za to uzyskamy równość
$K\left(\xi_2, \mu_{k+1}\right)
= K\left(\xi_1, \mu_k\right)
= K\left(\mu_1, \mu_{k+1}\right)$. Wynika to z tego, iż
$
K\left(\xi_2, \mu_{k}\right) \left(\mu_{k+1}\right),
K\left(\xi_2, \mu_{k}\right) \left(\xi_{1}\right)$ oba są rozszerzeniami
$p$-tego stopnia i pierwsze ciało zawiera się w drugim.

Niech końcówka słowa $w$ ma postać
\begin{equation}
w = \ldots g^{-1}r_lfr_1, l \geq 1
\label{eq:h_2_w_ending}
\end{equation}
$f, g$ są słowami wielomianowymi zaczynającymi się od translacji i $g^{-1}$
oznacza odwrotność $g$. Niech $u = \left(s_1 \ldots s_{k-1}\right)^{-1}$. Jest
jasne, że $u$ ma tę samą długość co $s_1\ldots s_{k-1}$ mniejszą od $n$ i kończy
się $r_1$. W szczególności $u\left(\mu_k\right) = \xi_1$ zatem $u = \left(\mu_k,
\xi_1\right)$. Oznaczmy sobie ciąg zmiennych powstałych przez kolejne aplikacje
sylab z $u$ jako $\lambda_1 = \mu_k, \lambda_2 =
s_{k-1}^{-1}\left(\lambda_1\right), \ldots$.

Niech $F = K \left(\mu_k, \mu_{k+1}\right)$. Mamy, że
$\mu_{k+1}^p = f\left(\mu_k\right)$ zatem $F$ jest prostym rozszerzeniem
$K\left(\lambda_1 \right)$ stopnia $p$ zawartym w $K \left( \mu_k, \xi_1
\right)$, ale nie $K \left(\mu, \lambda_2 \right)$.
Aplikacja $H_3(n)$ do $u$ daje nam, że $F \subseteq K \left( \lambda_1,
\lambda_2 \right)$.

Załóżmy, że $u$ składa się z więcej niż jednej sylaby. Aplikacja $H_1(n)$ do $u$
daje nam
$K \left( \lambda_1, \lambda_2 \right) \subseteq K \left( \lambda_1, \xi_2
\right) = K \left( \mu_k, \xi_2 \right)$, które dają $\mu_{k+1} \in F \subseteq
K \left( \xi_2, \mu_k \right)$.

Zatem $u$ jest monosylabowe i $\xi_1^p \in K \left( \mu_k\right)$. Zatem
$K \left( \mu_k \right) \left(\xi_1 \right) = K \left(\mu_k \right)
\left(\mu_{k+1} \right) = F$. Zatem całe słowo $w$ jest postaci
\[w = e_1 g^{-1}r_lfr_1\]
Dla pewnego $t$ względnie pierwszego z $p$ mamy $\mu_{k+1} \mu_1^t \in K \left(
\mu_k \right)$.  Podnosząc to do $p$-tej potęgi i podstawiając $x$ za $\mu_k$
otrzymamy

\[f(x)g^t \left( x^{p^l} \right) = h^p(x)\]

Sprzeczność z lematem \ref{lem:main_polynomial_lemma}.

\section{Dowód $H_3(n+1)$}
Niech $F$ będzie prostym rozszerzeniem stopnia $p$ ciała $K_1$. Niech $F
\subseteq K_{k+1}$ i $F \not \subseteq K_2$. Zakładamy, że $F \not \subseteq
K_k$, gdyż w przeciwnym wypadku z $H_3(n)$ mielibyśmy $F \subseteq K_2$.
Podobnie zakładamy, że $w$ kończy się pierwiastkiem $r_{m_k}, m_k \geq 1$.

Niech $F_1 = F \left( \mu_k \right)$. $K_k \subseteq F_1$, gdyż $\mu_1, \mu_k
\in F_1$. $F_1$ jest prostym rozszerzeniem $K_k$ stopnia $p$, gdyż
\[
\left[F_1:F\right]\left[F:K_1 \right] =
\left[F_1:K_k\right]\left[K_k:K_1\right]
\]
i $\left[F_1:F\right] \leq \left[K_k:K_1\right]$ (przypominamy, że $F_1 =
F\left(\mu_k\right), K_k = K_1\left(\mu_k\right)$ i $F_1 \supseteq K_k, F
\supsetneq K_1$).
Wiemy już z $H_1(n+1), H_2(n+1)$, że $K_{k+1}$ jest prostym rozszerzeniem $K_k$
stopnia $p^{m^k}$, zatem $F_1 = K_k \left(\mu_{k+1}^{p^{m_k -1}} \right)$.
Możemy $\mu_{k+1}$ dostosować tak by $m_k = 1$ i $F_1 = K_{k+1}$.

Zapisujemy końcówkę słowa $w$ podobnie jak w \ref{eq:h_2_w_ending}. Niech $F_0 =
F \left( \mu_{k-1} \right)$, podciało $F_1$. Naturalnie $F_0 \supseteq K_{k-1}$
i podobnie jak w przypadku $F_1$: $\left[F_0:K_{k-1} \right] = p$. Z drugiej
strony $\left[ K_{k+1} : K_{k-1} \right] = p^{l+1}$.  Zatem $F_1 = F_0
\left(\mu_k \right)$ jest prostym rozszerzeniem $F_0$ o stopniu $p^l$, gdyż
\[
  \left[K_{k+1} : F_0\right]\left[F_0:K_{k-1} \right] =
  \left[F_{k+1}:K_k\right]\left[K_k:K_{k-1}\right]
\]

Niech, więc $\tau$ będzie $F_0$-automorfizmem ciała $K_{k+1}$, które odwzorowuje
$\mu_k \rightarrow \omega \mu_k$, gdzie $\omega \neq 1$ jest $p$-tym
pierwiastkiem jedności. Niech $\overline{\mu}_{k+1} = \tau \left( \mu_{k+1}
\right) \in K_{k+1}$. Zatem $\overline{\mu}_{k+1}^p = f \left( \omega \mu_k
\right) \in K \left( \mu_k \right)$.
Zatem $K_k \left(\overline{\mu}_{k+1} \right)$ jest prostym rozszerzeniem $K_k$
stopnia $p$ zawartym w $K_{k+1}$, więc musi być identyczne z $K_{k+1}$.

Z lematu \ref{lem:associated_extensions} istnieje $t$ takie, że 
$p \! \not \!\, | \; t$ i
$\mu_{k+1} \overline{\mu}_{k+1}^t \in K_k$. Dodatkowo
\[ K \left( \mu_k, \mu_{k+1} \overline{\mu}_{k+1}^t \right)
  \subseteq
  K \left( \mu_k, \mu_1 \right)
\]
Jednakże $\left(\mu_{k+1}\overline{\mu}_{k+1}^t\right)^p =
f\left(\mu_k\right)f^t\left(\omega \mu_k\right) \in K\left(\mu_k\right)$. Zatem
$K \left( \mu_k, \mu_{k+1} \overline{\mu}_{k+1}^t \right)$
jest prostym rozszerzeniem $K\left(\mu_k\right)$ stopnia $p$ zawartym w
$K_k$. Podobnie jak w poprzedniej sekcji, aplikacja $H_3(n)$ do $u = \left(s_1
\ldots s_{k-1}\right)^{-1}$ z $\lambda_i$ zdefiniowanymi jak poprzednio daje
\[ \mu_{k+1}\overline{\mu}^t_{k+1} \in K\left(\lambda_1, \lambda_2\right) =
K\left(\mu_k, \lambda_2\right)\]

Niech $\left[K\left(\mu_k\right)\left(\lambda_2\right) :
K\left(\mu_k\right)\right] = p^m$.
Wówczas 
$K\left(\mu_k\right)\left(\lambda_2^{p^{m-1}}\right)/K\left(\mu_k\right)$ jest
prostym rozszerzeniem stopnia $p$ zawartym w $K_k$, czyli jest równe 
$K \left( \mu_k, \mu_{k+1} \overline{\mu}_{k+1}^t \right)$.
Mamy zatem dla pewnego $s$ niepodzielnego przez $p$:
\[\mu_{k+1} \overline{\mu}_{k+1}^tv_2^s \in K\left(\mu_k\right)\]
Wzięcie $p$-tych potęg i zastąpienie $\mu_k$ przez $x$ daje
\[f(x)f^t(\omega x)g^s\left(x^{p^l}\right) = h^p(x)\]
dla pewnego wielomianu $h$. Sprzeczność z lematem
\ref{lem:main_polynomial_lemma}.
