\chapter{Twierdzenie White'a}

\section{Wstęp}

Niech $p$ będzie dowolną nieparzystą liczbą pierwszą. Rozważmy permutacje $t :
x \rightarrow  x+1$ i $e: x \rightarrow x^p$ zbioru liczb rzeczywistych. W 1986
Samuel White \cite{whi88} udowodnił, że dla dowolnych liczb liczb całkowitych
$a_1, \ldots, a_n, k_1, \ldots, k_n$, gdzie $a_2, \ldots, a_n, k_1, \ldots,
k_{n-1}$ są niezerowe, odwzorowanie 

\begin{equation}
  t^{a_1}e^{k_1}\ldots t^{a_n}e^{k_n}
\label{eq:gp_element}
\end{equation}

nie jest identycznością.  Innymi słowy, podgrupa $G_p$ grupy $\Sigma_\R$
generowana przez $t, e$ jest wolna.

Słowa zredukowane postaci \ref{eq:gp_element}, gdzie wykładniki $k_i$ są
nieujemne, będziemy nazywać \textbf{słowami wielomianowymi}.

W tym rozdziale przedstawimy dowód tego twierdzenia oraz faktów pomocniczych.

\begin{defin}[Rozszerzenie czysto przestępne]
  Rozszerzenie $L$ ciała $K$ nazywamy czysto przestępnym wtw., gdy istnieje
  zbiór algebraicznie niezależny $A \subseteq L$ taki, że $K(A) = L$. Równoważne
  gdy $L$ jest izomorficzne z ciałem funkcji wymiernych w ciele $K$ nad
  zmiennymi $\left\{X_a : a \in A\right\}$.
\end{defin}

\section{Przestępny łańcuch}
\label{sec:przelan}

Niech $K = \bar{\Q}$ - domknięcie ciała liczb wymiernych, czyli zbiór wszystkich
liczb algebraicznych.

Niech $\xi$ to będzie dowolna liczba przestępna. Potrzebujemy zdefiniować ciąg
liczb przestępnych będących wynikiem kolejnych aplikacji liter tworzących słowo
$w$ na $\xi$. W tym celu oznaczmy sobie możliwe elementy. Niech $e_m: x
\rightarrow x^{p^m}, m \geq 1$ a jej odwrotność to $r_m: x \rightarrow
x^{\frac{1}{p^m}}, m \geq 1$. Translację oznaczymy jako $t_a: x \rightarrow x +
a, a \in \N, a \neq 0$.  Wówczas słowo $w$ to ciąg symboli postaci $w =
v_1\ldots v_n$, gdzie $v_i$ to $t_a$, $e_m$, lub $r_m$. 

Definiujemy $n+1$-elementowy ciąg $\left(\xi_i\right)$ jako: $\xi_1 = \xi,
\xi_{j+1} = \xi_{j}v_j, j \in \{1, \ldots, n\}$. W szczególności $\xi_{n+1} =
w(\xi)$.

Rozważmy teraz ciąg rozszerzeń ciała $K$:
$
K\left(\xi_1\right),
K\left(\xi_1, \xi_2\right),
K\left(\xi_1, \xi_3\right),
\ldots,
K\left(\xi_1, \xi_{n+1}\right)$.

Twierdzę, że ten ciąg tworzy wieżę:

\begin{equation}
K\left(\xi_1\right) \subseteq
K\left(\xi_1, \xi_2\right) \subseteq
K\left(\xi_1, \xi_3\right) \subseteq 
\ldots
\subseteq
K\left(\xi_1, \xi_{n+1}\right)
\label{eq:tower}
\end{equation}

Załóżmy, że \ref{eq:tower} zostało udowodnione. Z tego ciągu wynika nierówność
$w\left(\xi\right) \neq \xi$. Aby to zauważyć wystarczy rozważyć dwa przypadki.
Jeśli wśród $v_i$ nie ma pierwiastków to $w$ jest wielomianem nad $\Q$. Wówczas
$w(\xi) = \xi$ jest sprzeczne z przestępnością $\xi$ zatem $w(\xi) \neq \xi$.
Niech, więc $v_j = r_m$ będzie pierwszym pierwiastkiem wśród napisu $w$. Wówczas
$\xi_j \in K\left(\xi_1\right)$ i pokażę również, że $\xi_{j}^{\frac{1}{p}} \not
\in K(\xi_1)$. Zatem $K\left(\xi_1\right) \subset K\left(\xi_1,
\xi_{j+1}\right)$. Z \ref{eq:tower} mamy zatem, że $K\left(\xi_1\right) \neq
  K\left(\xi_1, \xi_{n+1}\right)$, czyli $w(\xi) \neq \xi$.

Dowód \ref{eq:tower} zostanie przeprowadzony indukcyjne, równocześnie z dwoma
innymi własnościami.

\section{Podłańcuch}

Zamiast rozpatrywać $w$ jako napis symboli $v_i$, zbierzemy je w sylaby: $s_1,
\ldots, s_k$, gdzie każda sylaba kończy się pierwiastkiem $r_{m_j}$. Niech słowo
$f \in G_p$ będzie słowem wielomianowym, jeśli $f$ składa się tylko z
translacji $t_a$ i dodatnich potęg $e_m$, czyli $f(x)$ jest po prostu
wielomianem. Możemy wówczas napisać, że $w = s_1\ldots s_k$, gdzie $s_j =
f_jr_{m_j}$, $f_j$ jest $j$-tym słowem wielomianowym. Aby zachować proste
oznaczenia pozwalamy na to by $f_1 = x$  i $s_k = f_k$, w tym przypadku będziemy
przyjmowali, że $m_k = 0$. 

Podobnie jak w 
\ref{sec:przelan} definiujemy podłańcuch $\left\{\mu_1, \ldots,
\mu_{k+1}\right\}$ jako $\mu_1 = \xi_1 = \xi, \mu_{j+1} = s_j\left(\mu_j\right)$
dla $1 \leq j \leq k$. Wówczas $\mu_{k+1} = \xi_{n+1} = w(\xi)$.

Zauważmy, że jeśli zachodzi \ref{eq:tower} to zachodzi też następujące:

\[
  K\left(\mu_1\right) \subseteq
  K\left(\mu_1, \mu_2\right) \subseteq
  \ldots \subset
  K\left(\mu_1, \mu_{k+1}\right)
\]

Pokazaliśmy już, że pierwsze zawieranie jest ścisłe. Drugą rzeczą, którą
będziemy dowodzić jest właśnie:

\begin{equation}
  K(\mu_1) \subset
  K(\mu_1, \mu_2) \subset
  \ldots \subset
  K(\mu_1, \mu_{k+1})
  \label{eq:strict_tower}
\end{equation}

Jeśli $m_k = 0$ to ostatnie zawieranie nie jest ścisłe.

[TODO czy zawieranie ścisłe to właściwy zwrot]

Do trzeciej własności zauważmy, że $\mu_2^{p^{m_1}} = f_1\left(\mu_1\right)$.
Zatem $K\left(\mu_1\right)\left(\sqrt[p^{m_1 -
1}]{f_1\left(\mu_1\right)}\right)$ jest rozszerzeniem prostym
$K\left(\mu_1\right)$ o stopniu $p$.
[TODO w pracy tutaj jest pierwiastek stopnia p i rozszerzenia stopnia p. Czy to
jest błąd?]
To rozszerzenie jest zawarte w $K\left(\mu_1, \mu_{k+1}\right)$. Czy może
istnieć inne rozszerzenie proste o tej własności? Okazuje się, że nie.

\section{Rezultat}

Będziemy dowodzić 3 fakty równoczesną indukcyjną po długości $n$ słowa $w$.

Oznaczmy $K\left(\mu_1, \mu_j\right)$ jako $K_j$.

\begin{theorem}
  Niech $w = v_1 \ldots v_n = s_1 \ldots s_k$ będzie słowem w $G_p$ i $\xi$
  rzeczywistą liczbą przestępną. Wówczas

  \begin{description}
    \item{$H_1$:} 
      $K\left(\xi_1\right)\subseteq
      K\left(\xi_1, \xi_2\right)\subseteq
      \ldots \subseteq
      K\left(\xi_1, \xi_{n+1}\right)$
    \item{$H_2$:}
      $K_1 \subset K_2 \subset \ldots \subset K_{k+1}$ (ostatnie zawieranie jest
      równością, jeśli $m_k = 0$)
    \item{$H_3$:}
      Jeśli $F$ jest prostym rozszerzeniem $K_1$ rzędu $p$  zawartym w $K_{k+1}$
      to $F \subseteq K_2$.
  \end{description}
  \label{th:hypothesis_h}
\end{theorem}

Pokazaliśmy już jak z \ref{th:hypothesis_h} wynika

\begin{corollary}[Twierdzenia White'a \cite{whi88}]
  Grupa $G_p$ jest grupą wolną rzędu 2.
\end{corollary}

Niech $H(n)$ oznacza prawdziwość Hipotezy H dla słów długości nieprzekraczającej
$n$.

Dla $n=1$ mamy:
\begin{description}
  \item{$H_1$:} 
    $K\left(\xi_1\right) \subseteq K\left(\xi_1, \xi_2\right)$, co jest
    oczywiste.
  \item{$H_2$:} Dla $m_1 = 0$ ta hipoteza nic nie stwierdza. W przeciwnym
    wypadku ta hipoteza stwierdza, że $K\left(\xi_1\right) \subset K\left(\xi_1,
    \xi_2\right)$. $\xi_2$ jest pierwiastkiem $\xi_1$ o stopniu podzielnym przez
    $p$, więc wykorzystując lemat \ref{lem:nontrivial_roots} i znany fakt
    $K\left(\xi_1\right) \cong K[x]$ mamy, że $\xi_2 \not \in
    K\left(\xi_1\right)$.
  \item{$H_3$:} Prawdziwe, bo $K_{k+1} = K_2$.

\end{description}

Zatem będziemy przyjmować, że $n \geq 2$. Ciąg wywodu będzie formować wg.
następującego schematu:

\[ H(n) \Rightarrow H_1(n+1) \Rightarrow H_2(n+1) \Rightarrow H_3(n+1)
\Rightarrow H(n+1)\]

Zauważmy, że jeśli słowo $w$ zaczyna się lub kończy przesunięciem to krok
indukcyjny jest łatwy do udowodnienia. Zaiste, jeśli $w$ kończy się translacją
to $H_2(n+1)$ trywialnie wynika z $H(n)$.  Z równości $K\left(\xi_1,
\xi_{n}\right) = K\left(\xi_1, \xi_{n} + a\right), \xi_{n+1} = \xi_n + a$ wynika
$H_1(n+1), H_3(n+1)$. Jeśli $w$ zaczyna się translacją to podobnie tylko tym
razem korzystamy z równości $K\left(\xi_1, \xi_1 + a\right) =
K\left(\xi_1\right)$. Zatem będziemy zakładać, że $w$ zaczyna i kończy się $e_m$
lub $r_m$ dla pewnego $m \geq 1$.

\section{Dowód $H_1(n+1)$}
Przyjmujemy notację i założenia z poprzednich sekcji oraz hipotezę $H(n)$.
Będziemy też używać notacji typu $\left(\xi_i, \xi_j\right)$ do oznaczenia
podsłowa $v_i\ldots v_{j-1}$ słowa $w$ którego działanie przekształca $\xi_i$ na
$\xi_j$. Dokonamy teraz dodatkowych obserwacji i poprawek.

Z $H_1(n)$ zaaplikowanego do $\left(\xi_1, \xi_{n+1}\right)$ oraz $\left(\xi_2,
\xi_{n+2}\right)$ mamy

\begin{equation}
  K\left(\xi_1\right) \subseteq
  K\left(\xi_1, \xi_2\right) \subseteq
  \ldots \subseteq
  K\left(\xi_1, \xi_{n+1}\right)
  \label{eq:h_1_first}
\end{equation}

\begin{equation}
  K\left(\xi_2\right) \subseteq
  K\left(\xi_2, \xi_3\right) \subseteq
  \ldots \subseteq
  K\left(\xi_2, \xi_{n+2}\right)
  \label{eq:h_1_second}
\end{equation}

Załóżmy, że $H_1(n+1)$ nie zachodzi, tj. $K\left(\xi_1, \xi_{n+2}\right)$ nie
zawiera się w $K\left(\xi_1, \xi_{n+1}\right)$. Wówczas $v_{n+1} = e_m$ i $m_k =
0$. 
Naturalnie $\xi_{n+2} = v_{n+1}\left(\xi_{n+1}\right) \in
K\left(\xi_{n+1}\right)$, czyli $K\left(\xi_1, \xi_{n+2}\right) \subset
K\left(\xi_1, \xi_{n+1}\right)$.
Zauważmy, że jeśli $\xi_2 \in K\left(\xi_1\right)$ to z ostatniego zawierania w
\ref{eq:h_1_second} mamy:
\[ 
  K\left(\xi_1, \xi_{n+1}\right) =
  K\left(\xi_2, \xi_{n+1}\right)\left(\xi_1\right) \subseteq
  K\left(\xi_2, \xi_{n+2}\right)\left(\xi_1\right) =
K\left(\xi_1, \xi_{n+2}\right)\]
Zatem musimy założyć, że $v_1 = r_{m_1}$.

Będziemy chcieli, aby zachodziło $K\left(\xi_1, \xi_{n+2}\right) \cap
K\left(\xi_1, \xi_2\right) = K\left(\xi_1\right)$. W tym celu podstawiamy pod
$\xi_1 = \xi_2^{p^{m_1-1}}$ i $v_1 = r_1$. To podstawienie daje nam tę własność,
gdyż $\xi_2 \not \in K\left(\xi_2^{p^{m_1}}\right)$ i stopień rozszerzenia
$\left[K_2 : K_1\right] = p$. To podstawienie nie osłabia nam dowodzonej
własności, gdyż jeśli oznaczymy podstawione $\xi_1$ jako $\xi_1^*$ to
$K\left(\xi_1\right) \subseteq K\left(\xi_1^*\right)$.
[TODO czemu tak chcemy?]

Dodatkowo zauważmy, że 
$\xi_{n+2} \in K\left(\xi_{n+1}\right) \land \xi_{1} \in K\left(\xi_{2}\right)$.
Zatem z \ref{eq:h_1_first} i \ref{eq:h_1_second} mamy
\[
  K\left(\xi_1, \xi_{n+1}\right) =
  K\left(\xi_2, \xi_{n+2}\right) =
  K\left(\xi_2, \xi_{n+1}\right)
\]

Używając sylab możemy zauważyć, że powyższe daje nam:
\[
  K\left(\mu_1, \mu_{k}\right) =
  K\left(\mu_2, \mu_{k+1}\right) =
  K\left(\mu_2, \mu_{k}\right)
\]
Jest to ciało, które ściśle zawiera $K_{k+1}$. Dodatkowo $K_{k+1} \cap K_2 =
K_1$ i $K_k = K_{k+1}\left(\mu_2\right)$ jest czystym rozszerzeniem $K_{k+1}$.
$K_j = K\left(\mu_2, \mu_j\right)$ dla $2 \leq j \leq k$, gdyż $\mu_1 \in
K\left(\mu_2\right)$ i $\mu_2 \in K_j$. Zatem możemy zauważyć, że $\left[K_k :
K_{k+1}\right] = p$.

Załóżmy teraz, że $k \geq 3$. $K_{k+1}$ jest rozszerzeniem $K_1$ zawartym w
$K_k$, ale to rozszerzenie niekoniecznie jest czyste.

Niech $\omega \neq 1$ będzie $p$-tym pierwiastkiem jedności zawartym w $K_k$.
Możemy zdefiniować $K_{k+1}$-automorfizm $\tau$ ciała $K_k$, który przekształca
$\mu_2$ na $\omega \mu_2$. 
Zauważmy, że $K_2 = K\left(\mu_2\right) = K\left(\omega \mu_2\right) =
\overline{K_2}$.
Niech $\overline{\mu_3} = \tau(\mu_3) \in K_k$ i niech druga sylaba $s_2$ słowa
$w$ będzie postaci $fr_l$.
Działanie $\tau$ na $\mu_3^{p^l} = f\left(\mu_2\right)$ daje
$\overline{\mu_3}^{p^l} = f\left(\omega\mu_2\right)$. Zatem $K_3$ i
$\overline{K_3} = K\left(\mu_2, \overline{mu_3}\right)$ są czystymi
rozszerzeniami stopnia $p_l$ ciała $K_2$ zawartego w $K_k$.

[TODO, czemu $K_k$ jest równe $\overline{K_k}$? Jaka jest fundamentalna różnica
między poprzednimi ciała, że takie coś zachodzi]

Jeśli $l = 1$ to użycie $H_3(n)$ do podsłowa $\left(\mu_2, \mu_k\right)$ daje
równość $\overline{K_3} = K_3$. Jeśli $l > 1$ to zauważmy, że jeśli $K_3 \cap
\overline{K_3} = K_2$ to $K_2\left(\overline{\mu_3}^{p^{l-1}}\right)$ jest
czystym rozszerzeniem $K\left(\mu_2\right)$ stopnia $p$ zawartym w $K_k$, ale
nie w $K_3$. Sprzeczność. Jeśli $K_3 \cap \overline{K_3} \supset K_2$ to
wystarczy odpowiednio dostosować $\mu_2$ i $l$ jak poprzednio.

Teraz z lematu TODO równość $K_2\left(\mu_3\right) =
K_2\left(\overline{\mu_3}\right)$ implikuje, że dla pewnego $t$ ($p \not | t$)
mamy $\mu_3\overline{\mu_3}^t \in K\left(\mu_2\right)$. Podnosząc to do
$p^l$-tej potęgi i zastępując $\mu_2$ przez $x$ otrzymamy równanie
\[ f(x)f^t(\omega x) = h^{p^l}(x)\]

To jest niemożliwe z lematu \ref{lem:main_polynomial_lemma}. Zatem $k = 2$ i $w
= r_{m_1}f$, gdzie $f$ jest słowem wielomianowym postaci $f_0^{p^l}(x)$, gdzie
$v_{n+1} = e_l$. Mamy zatem $K_3 = K_1$ [TODO czemu?] i  w takim razie $\mu_3 =
f\left(\mu_2\right) \in K_1 = K\left(\mu_2^p\right)$. To prowadzi do $f_0(x) =
g\left(x^p\right)$ i sprzeczności.

\section{Dowód $H_2(n+1)$}
Zakładamy $H_1(n+1), H(n)$. Będziemy używać założenie indukcyjne na odwrotności
podsłowa $w$. 

Mając $H_2(n)$ musimy udowodnić, że $K_k \subset K_{k+1}$, gdy $m_k \geq 1$.
Załóżmy, że $H_2(n+1)$ nie zachodzi, czyli $K_k = K_{k+1}$. To oznacza, że
$\mu_{k+1} \in K_k$, zatem naturalnie $\mu_{k+1}^{p^{m_k -1}} \in K_k$. Możemy
więc zastąpić $v_n = r_{m_k}$ przez $r_1$ a przez to poprzednie $\mu_{k+1}$
staje się równe $\mu_{k+1}^{p^{m_k - 1}}$.
Jeśli $\mu_{k+1} \in K\left(\xi_2, \mu_k\right)$ to $K\left(\xi_2, \mu_k\right)
= K\left(\xi_2, \mu_{k+1}\right)$ co zaprzecza $H_2(n)$ zaaplikowanej do
$\left(\xi_2, \xi_{n+2}\right)$. Zatem $\mu_{k+1} \not \in K\left(\xi_2,
\mu_k\right)$ i $w$ musi zaczynać się od potęgi $v_1 = e_q$.

Z hipotezy $H_1(n+1)$ mamy
\[ 
  K\left(\xi_2, \mu_k\right) \left(\mu_{k+1}\right)
  =
  K\left(\xi_2, \mu_{k+1}\right)
  \subseteq
  K\left(\xi_1, \mu_{k+1}\right)
  =
  K\left(\xi_1, \mu_{k}\right)
  =
  K\left(\xi_2, \mu_{k}\right)\left(\xi_1\right)
\]

Poprzez tą samą metodę co w poprzedniej sekcji dostowojemy tak $\xi_1$ aby móc
założyć $q = 1$. Wówczas dalej będzie $\xi_1 \not \in K\left(\xi_2,
\mu_k\right)$, ale za to uzyskamy równość 
$K\left(\xi_2, \mu_{k+1}\right) 
= K\left(\xi_1, \mu_k\right) 
= K\left(\mu_1, \mu_{k+1}\right)$. Wynika to z tego, iż 
$
K\left(\xi_2, \mu_{k}\right) \left(\mu_{k+1}\right),
K\left(\xi_2, \mu_{k}\right) \left(\xi_{1}\right)$ oba są rozszerzeniami
$p$-tego stopnia i pierwsze ciało zawiera się w drugim.

Niech końcówka słowa $w$ ma postać
\[ w = \ldots e_mg^{-1}r_lfr_1, m,l \geq 1\]

[TODO czemu $e_m$? to wcale nie jest konieczne]

$f, g$ są słowami wielomianowymi zaczynającymi się od translacji i $g^{-1}$
oznacza odwrotność $g$. Niech $u = \left(s_1 \ldots s_{k-1}\right)^{-1}$. Jest
jasne, że $u$ ma tę samą długość co $s_1\ldots s_{k-1}$ mniejszą od $n$ i kończy
się $r_1$. W szczególności $u\left(\mu_k\right) = \xi_1$ zatem $u = \left(\mu_k,
\xi_1\right)$. Oznaczmy sobie ciąg zmiennych powstałych przez kolejne aplikacje
sylab z $u$ jako $\lambda_1 = \mu_k, \lambda_2 =
s_{k-1}^{-1}\left(\lambda_1\right), \ldots$.

Niech $F = K \left(\mu_k, \mu_{k+1}\right)$. Mamy, że 
$\mu_{k+1}^p = f\left(\mu_k\right)$ zatem $F$ jest czystym rozszerzeniem
$K\left(\lambda_1 \right)$ stopnia $p$ zawartym w $K \left( \mu_k, \xi_1
\right)$, ale nie $K \left(\mu, \lambda_2 \right)$. 
Aplikacja $H_3(n)$ do $u$ daje nam, że $F \subseteq K \left( \lambda_1,
\lambda_2 \right)$.

Załóżmy, że $u$ składa się z więcej niż jednej sylaby. Aplikacja $H_1(n)$ do $u$
daje nam 
$K \left( \lambda_1, \lambda_2 \right) \subseteq K \left( \lambda_1, \xi_2
\right) = K \left( \mu_k, \xi_2 \right)$, które dają $\mu_{k+1} \in F \subseteq
K \left( \xi_2, \mu_k \right)$.

Zatem $u$ jest monosylabowe i $\xi_1^p \in K \left( \mu_k\right)$. Zatem 
$K \left( \mu_k \right) \left(\xi_1 \right) = K \left(\mu_k \right)
\left(\mu_{k+1} \right) = F$. Zatem całe słowo $w$ jest postaci
\[w = e_1 g^{-1}r_lfr_1\]
Dla pewnego $t$ względnie pierwszego z $p$ mamy $\mu_{k+1} \mu_1^t \in K \left(
\mu_k \right)$. 


\section{Dowód $H_3(n+1)$}
Niech $F$ będzie czystym rozszerzeniem stopnia $p$ ciała $K_1$. Niech $F
\subseteq K_{k+1}$ i $F \not \subseteq K_2$.

