\chapter{Twierdzenie White'a}

\section{Wstęp}

Niech $p$ będzie dowolną nieparzystą liczbą pierwszą. Rozważmy permutacje $t :
x \rightarrow  x+1$ i $e: x \rightarrow x^p$ zbioru liczb rzeczywistych. W 1986
Samuel White \cite{whi88} udowodnił, że dla dowolnych liczb liczb całkowitych
$a_1, \ldots, a_n, k_1, \ldots, k_n$, gdzie $a_2, \ldots, a_n, k_1, \ldots,
k_{n-1}$ są niezerowe, odwzorowanie 

\begin{equation}
  t^{a_1}e^{k_1}\ldots t^{a_n}e^{k_n}
\label{eq:gp_element}
\end{equation}

nie jest identycznością.  Innymi słowy, podgrupa $G_p$ grupy $\Sigma_\R$
generowana przez $t, e$ jest wolna.

Słowa zredukowane postaci \ref{eq:gp_element}, gdzie wykładniki $k_i$ są
nieujemne, będziemy nazywać \textbf{słowami wielomianowymi}.

W tym rozdziale przedstawimy dowód tego twierdzenia oraz faktów pomocniczych.

\section{Przestępny łańcuch}
\label{sec:przelan}

Niech $K = \bar{\Q}$ - domknięcie ciała liczb wymiernych, czyli zbiór wszystkich
liczb algebraicznych.

Niech $\xi$ to będzie dowolna liczba przestępna. Potrzebujemy zdefiniować ciąg
liczb przestępnych będących wynikiem kolejnych aplikacji liter tworzących słowo
$w$ na $\xi$. W tym celu oznaczmy sobie możliwe elementy. Niech $e_m: x
\rightarrow x^{p^m}, m \geq 1$ a jej odwrotność to $r_m: x \rightarrow
x^{\frac{1}{p^m}}, m \geq 1$. Translację oznaczymy jako $t_a: x \rightarrow x +
a, a \in \N, a \neq 0$.  Wówczas słowo $w$ to ciąg symboli postaci $w =
v_1\ldots v_n$, gdzie $v_i$ to $t_a$, $e_m$, lub $r_m$. 

Definiujemy $n+1$-elementowy ciąg $\left(\xi_i\right)$ jako: $\xi_1 = \xi,
\xi_{j+1} = \xi_{j}v_j, j \in \{1, \ldots, n\}$. W szczególności $\xi_{n+1} =
w(\xi)$.

Rozważmy teraz ciąg rozszerzeń ciała $K$:
$
K\left(\xi_1\right),
K\left(\xi_1, \xi_2\right),
K\left(\xi_1, \xi_3\right),
\ldots,
K\left(\xi_1, \xi_{n+1}\right)$.

Twierdzę, że ten ciąg tworzy wieżę:

\begin{equation}
K\left(\xi_1\right) \subseteq
K\left(\xi_1, \xi_2\right) \subseteq
K\left(\xi_1, \xi_3\right) \subseteq 
\ldots
\subseteq
K\left(\xi_1, \xi_{n+1}\right)
\label{eq:tower}
\end{equation}

Załóżmy, że \ref{eq:tower} zostało udowodnione. Z tego ciągu wynika nierówność
$w\left(\xi\right) \neq \xi$. Aby to zauważyć wystarczy rozważyć dwa przypadki.
Jeśli wśród $v_i$ nie ma pierwiastków to $w$ jest wielomianem nad $\Q$. Wówczas
$w(\xi) = \xi$ jest sprzeczne z przestępnością $\xi$ zatem $w(\xi) \neq \xi$.
Niech, więc $v_j = r_m$ będzie pierwszym pierwiastkiem wśród napisu $w$. Wówczas
$\xi_j \in K\left(\xi_1\right)$ i pokażę również, że $\xi_{j}^{\frac{1}{p}} \not
\in K(\xi_1)$. Zatem $K\left(\xi_1\right) \subset K\left(\xi_1,
\xi_{j+1}\right)$. Z \ref{eq:tower} mamy zatem, że $K\left(\xi_1\right) \neq
  K\left(\xi_1, \xi_{n+1}\right)$, czyli $w(\xi) \neq \xi$.

Dowód \ref{eq:tower} zostanie przeprowadzony indukcyjne, równocześnie z dwoma
innymi własnościami.

\section{Podłańcuch}

Zamiast rozpatrywać $w$ jako napis symboli $v_i$, zbierzemy je w sylaby: $s_1,
\ldots, s_k$, gdzie każda sylaba kończy się pierwiastkiem $r_{m_j}$. Niech słowo
$f \in G_p$ będzie słowiem wielomianowym, jeśli $f$ składa się tylko z
translacji $t_a$ i dodatnich potęg $e_m$, czyli $f(x)$ jest po prostu
wielomianem. Możemy wówczas napisać, że $w = s_1\ldots s_k$, gdzie $s_j =
f_jr_{m_j}$, $f_j$ jest $j$-tym słowem wielomianowym. Aby zachować proste
oznaczenia pozwalamy na to by $f_1 = x$  i $s_k = f_k$, w tym przypadku będziemy
przyjmowali, że $m_k = 0$. 

Podobnie jak w 
\ref{sec:przelan} definiujemy podłańcuch $\left\{\mu_1, \ldots,
\mu_{k+1}\right\}$ jako $\mu_1 = \xi_1 = \xi, \mu_{j+1} = s_j\left(\mu_j\right)$
dla $1 \leq j \leq k$. Wówczas $\mu_{k+1} = \xi_{n+1} = w(\xi)$.

Zauważmy, że jeśli zachodzi \ref{eq:tower} to zachodzi też następujące:

\[
  K\left(\mu_1\right) \subseteq
  K\left(\mu_1, \mu_2\right) \subseteq
  \ldots \subset
  K\left(\mu_1, \mu_{k+1}\right)
\]

Pokazaliśmy już, że pierwsze zawieranie jest ścisłe. Drugą rzeczą, którą
będziemy dowodzić jest właśnie:

\begin{equation}
  K(\mu_1) \subset
  K(\mu_1, \mu_2) \subset
  \ldots \subset
  K(\mu_1, \mu_{k+1})
  \label{eq:strict_tower}
\end{equation}

Jeśli $m_k = 0$ to ostatnie zawieranie nie jest ścisłe.

[TODO czy zawieranie ścisłe to właściwy zwrot]

Do trzeciej własności zauważmy, że $\mu_2^{p^{m_1}} = f_1\left(\mu_1\right)$.
Zatem $K\left(\mu_1\right)\left(\sqrt[p^{m_1 -
1}]{f_1\left(\mu_1\right)}\right)$ jest rozszerzeniem prostym
$K\left(\mu_1\right)$ o stopniu $p$.
[TODO w pracy tutaj jest pierwiastek stopnia p i rozszerzenia stopnia p. Czy to
jest błąd?]
To rozszerzenie jest zawarte w $K\left(\mu_1, \mu_{k+1}\right)$. Czy może
istnieć inne rozszerzenie proste o tej własności? Okazuje się, że nie.

\section{Rezultat}

Będziemy dowodzić 3 fakty równoczesną indukcyjną po długości $n$ słowa $w$.

Oznaczmy $K\left(\mu_1, \mu_j\right)$ jako $K_j$.

\begin{theorem}
  Niech $w = v_1 \ldots v_n = s_1 \ldots s_k$ będzie słowem w $G_p$ i $\xi$
  rzeczywistą liczbą przestępną. Wówczas

  \begin{description}
    \item{$H_1$:} 
      $K\left(\xi_1\right)\subseteq
      K\left(\xi_1, \xi_2\right)\subseteq
      \ldots \subseteq
      K\left(\xi_1, \xi_{n+1}\right)$
    \item{$H_2$:}
      $K_1 \subset K_2 \subset \ldots \subset K_{k+1}$ (ostatnie zawieranie jest
      równością, jeśli $m_k = 0$)
    \item{$H_3$:}
      Jeśli $F$ jest prostym rozszerzeniem $K_1$ rzędu $p$  zawartym w $K_{k+1}$
      to $F \subseteq K_2$.
  \end{description}
  \label{th:hypothesis_h}
\end{theorem}

Pokazaliśmy już jak z \ref{th:hypothesis_h} wynika

\begin{corollary}[Twierdzenia White'a \cite{whi88}]
  Grupa $G_p$ jest grupą wolną rzędu 2.
\end{corollary}

Niech $H(n)$ oznacza prawdziwość Hipotezy H dla słów długości nieprzekraczającej
$n$.

Dla $n=1$ mamy:
\begin{description}
  \item{$H_1$:} 
    $K\left(\xi_1\right) \subseteq K\left(\xi_1, \xi_2\right)$, co jest
    oczywiste.
  \item{$H_2$:} Dla $m_1 = 0$ ta hipoteza nic nie stwierdza. W przeciwnym
    wypadku ta hipoteza stwierdza, że $K\left(\xi_1\right) \subset K\left(\xi_1,
    \xi_2\right)$. $\xi_2$ jest pierwiastkiem $\xi_1$ o stopniu podzielnym przez
    $p$, wieć wykorzystując lemat \ref{lem:nontrivial_roots} i znany fakt
    $K\left(\xi_1\right) \cong K[x]$ mamy, że $\xi_2 \not \in
    K\left(\xi_1\right)$.
  \item{$H_3$:} Prawdziwe, bo $K_{k+1} = K_2$.

\end{description}

Zatem będziemy przyjmować, że $n \geq 2$. Ciąg wywodu będzie formować wg.
następującego schematu:

\[ H(n) \Rightarrow H_1(n+1) \Rightarrow H_2(n+1) \Rightarrow H_3(n+1)
\Rightarrow H(n+1)\]

Zauważmy, że jeśli słowo $w$ zaczyna się lub kończy przesunięciem to krok
indukcyjny jest łatwy do udowodnienia. Zaiste [TODO]
