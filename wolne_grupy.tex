\chapter{Grupy wolne}
\label{ch:free_groups}
Podstawową ideą stojąca za pojęciem grupy wolnej $F(S)$ generowanej przez zbiór
$S$ jest to, że nie istnieje żadna nietrywialna relacja między elementami $S$ w
tej grupie. Nie istnieje żaden nietrywialny sposób uproszczenia ciągu operacji
na $S$.
Formalnie grupę wolną definiujemy następująco:

\begin{defin}[Uniwersalna własność grup wolnych]
  Niech $S$ będzie zbiorem a $F$ grupą taką, że istnieje odwzorowanie $\iota: S
  \rightarrow F$. $F$ nazywa się grupą wolną generowaną przez zbiór $S$ jeśli
  dla każdej grupy $G$ i odwzorowania $\rho: S \rightarrow F$ istnieje dokładnie
  jeden homomorfizm $\phi: F \rightarrow G$ taki, że:

  \begin{equation}
  \forall_{s \in S} \,\rho(s) = (\phi \circ \iota) (s)
    \label{eq:universal_property}
  \end{equation}
  \label{def:universal_property}
\end{defin}

Z tej własności można udowodnić, że $\iota$ jest iniekcją i obraz $\iota$
generuje $F$.

\begin{corollary}
  $\iota$ jest iniekcją.
\end{corollary}
\begin{proof}
  Jeśli $|S| < 2$ to różnowartościowość trywialnie zachodzi. W przeciwnym
  wypadku niech $a, b \in S, a \neq b$. Niech $\rho: S \rightarrow \Z_2, \rho(a)
  = 0, \rho(b) = 1$. Jeśli $\iota(a) = \iota(b)$ to $\phi(\iota(a)) =
  \phi(\iota(b))$, ale $\rho(a) \neq \rho(b)$. Sprzeczność.
\end{proof}

\begin{corollary}
  Obraz $\iota$ generuje $F$.
\end{corollary}
\begin{proof}
  Niech $G = \langle \iota(S) \rangle, \rho(s) = \iota(s)$.
  Istnieje unikalny homomorfizm $\phi: F \rightarrow \langle \iota(S) \rangle$
  spełniający \ref{eq:universal_property}.

  Ponieważ $\langle \iota(s) \rangle \subseteq F$ zatem istnieją co najmniej dwa
  możliwe homomorfizmy $\psi, \psi': F \rightarrow F$: $\psi = id$ i
  $\psi' = \phi$.

  Jeśli weźmiemy $F$ za $G$ i użyjemy uniwersalnej własności to z
  jednoznaczności $\phi$ otrzymamy, że $\psi = \psi'$, czyli $\phi = id$. Zatem
  $F = \langle \iota(S)\rangle$.
\end{proof}

\subsubsection{Dowód istnienia grupy wolnej generowanej przez zbiór $S$}
Mając dany zbiór $S$ możliwe jest skonstruowanie grupy wolnej $F(S)$ generowanej
przez ten zbiór.
Kluczowym pomysłem jest tutaj fakt, że elementy $F(S)$ muszą w pewien sposób
zakodować operacje użyte do wygenerowania podgrupy $\langle \rho(S) \rangle$,
ponieważ dla przykładu

\[
\phi\left(\iota\left(s_1\right) \ldots \iota\left(s_n\right)\right)
=
\phi\left(\iota\left(s_1\right)\right) \ldots \phi\left(\iota\left(s_n\right)\right)
=
\rho\left(s_1\right)\ldots\rho\left(s_n\right)\]

To powinno być możliwe dla każdej grupy $G$ dlatego potrzebujemy słów kodujących
elementy $s_i$.

\begin{defin}[Zbiór słów zredukowanych]
  Niech $S^{-1}$ będzie zbiorem takim, że $S \cap S^{-1} = \emptyset$ i istnieje
  bijekcja między $S$ i $S^{-1}$.
  Element zbioru $S^{-1}$ odpowiadający w tej bijekcji elementowi $s \in S$
  oznaczamy przez $s^{-1}$ i nazywamy elementem odwrotnym do $s$.
  Załóżmy dodatkowo, że $1 \not \in S \cup S^{-1}$ i $1^{-1} = 1$.
  Zbiorem słów zredukowanych $F(S)$ jest zbiór nieskończonych ciągów postaci:

  \[(s_1, s_2, \ldots)\]

  takich, że $s_i \in S \cup S^{-1} \cup \{1\}, s_i = 1$ for dostatecznie dużych
  $i$ oraz

\begin{enumerate}
  \item $s_{i+1} \neq s_i^{-1}$ dla każdego $i$ dla którego $s_i \neq 1$.
  \item Jeśli $s_k = 1$ to $\forall_{i \geq k} s_i = 1$
\end{enumerate}
\end{defin}

Słowo $(1,1,\ldots)$ nazywamy słowem pustym i będziemy oznaczać symbolem $1$.

Słowo zredukowane $
(s_1^{\epsilon_1},
s_2^{\epsilon_2}, \ldots,
s_n^{\epsilon_n}, 1, 1, \ldots)$,
gdzie $\epsilon_i \in \{-1, 1\}$ będziemy oznaczać napisem
$s_1^{\epsilon_1}\ldots s_n^{\epsilon_n}$.

\begin{defin}[Grupa wolna generowana przez zbiór]
  Grupą wolną generowaną przez zbiór $S$ nazywamy zbiór słów zredukowanych
  $F(S)$ gdzie operacją grupową $\cdot$ jest konkatenacja z eliminacją
  redukujących się symboli, tj. ciągów postaci $ss^{-1}$. Formalnie niech:
  $s = s_1^{\epsilon_1}\ldots s_n^{\epsilon_n}, p = p_1^{\epsilon_1}\ldots
  p_m^{\epsilon_n} \in F(S)$, $s_i, p_j \neq 1$ i $k \in \N$ jest takim
  największym takim indeksem, że:

  \[\forall_{l \leq k} \; s_{n - l + 1}^{\epsilon_{n - l + 1}} =
    p_l^{-\epsilon_l}\]

  Wówczas $s \cdot p = s_1^{\epsilon_1}\ldots s_{n -
  k}^{\epsilon_{n-k}}p_{k+1}^{\epsilon_{k+1}}\ldots p_{m}^{\epsilon_m}$. Jeśli
  $k$ jest równe długości któregoś ze słów to w powyższej definicji to słowo
  redukuje się do $1$.

  Iniekcję $\iota$ z definicji \ref{def:universal_property} definiujemy w
  następujący sposób: $\iota: S \rightarrow F(S), \iota(s) = s$.
  \label{def:generated_free_group}
\end{defin}

\begin{theorem}
  Grupa $F(S)$ z definicji \ref{def:generated_free_group} jest rzeczywiście grupą
  i dodatkowo jest grupą wolną generowaną przez $S$.
\end{theorem}
\begin{proof}
  Z definicji działania grupowego: $1 \cdot s = s \cdot 1 = s$ oraz
  dla każdego słowa
  $s = s_1^{\epsilon_1}\ldots s_n^{\epsilon_n}$
  jego odwrotnością jest:
  $s = s_n^{-\epsilon_n}\ldots s_1^{-\epsilon_1}$. Pozostaje tylko sprawdzić, że
  $\cdot$ jest operacją łączną. Dowód tej własności można przeprowadzić
  indukcyjnie, tutaj jednak użyję innego rozumowania.

  Zdefiniujmy $\sigma_s: F(S) \rightarrow F(S), s \in S \cup S^{-1} \cup \{1\}$,
  jako $\sigma_s(a) = \iota(s) \cdot a$. $\sigma_{s^{-1}} \circ \sigma_{s}$ jest
  identycznością zatem $\sigma_{s}$ jest permutacją na $F(S)$. Niech
  $A(S) = \left\langle \sigma_s : s \in S \cup S^{-1} \cup \{1\}\right\rangle$
  będzie podgrupą grupy symetrii $F(S)$.

  Odwzorowanie:
  \[
    s_1^{\epsilon_1} \ldots
    s_n^{\epsilon_n} \rightarrow
    \sigma_{s_1^{\epsilon_1}} \ldots
    \sigma_{s_n^{\epsilon_n}}
  \]

  jest bijekcją z $F(S)$ do $A(S)$ która zachowuje działanie grupowe obu
  zbiorów.
  Ponieważ działanie w $A(S)$ jest łączne zatem działanie $\cdot$ w
  $F(S)$ też musi być.

  Aby udowodnić wolność niech $G$ będzie dowolną grupą i $\rho: S \rightarrow G$
  dowolnym odwzorowaniem. Homomorfizm $\phi: F(S) \rightarrow G$ musi spełniać

  \[ \phi\left(s_1^{\epsilon_1} \ldots s_n^{\epsilon_n}\right) =
  \rho\left(s_1\right)^{\epsilon_1} \ldots \rho\left(s_n\right)^{\epsilon_n}\]

  To służy za definicję $\phi$ i pokazuje jej istnienie oraz jednoznaczność.
\end{proof}

\section{Unikalność grup wolnych}
Grupa słów zredukowanych jest utożsamiana z pojęciem grupy wolnej nad $S$ z
powodu następującego twierdzenia.

\begin{theorem}
  Jeśli $F, F'$ są grupami wolnymi nad $S$ to $F \cong F'$.
\end{theorem}
\begin{proof}
  Z definicji \ref{def:universal_property} istnieją jednoznaczne homomorfizmy
  $\phi: F \rightarrow F', \phi': F' \rightarrow F$, które są identycznością na
  zbiorze $S$. Zatem $\phi' \circ \phi: F \rightarrow F$ jest identycznością,
  gdyż $\iota(S)$ generuje zarówno $F$ jak i $F'$. Podobnie $\phi \circ \phi'$
  Zatem $\phi, \phi'$ są izomorfizmami.
\end{proof}
